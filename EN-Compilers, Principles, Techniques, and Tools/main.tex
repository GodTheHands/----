\documentclass[12pt,a4paper,twoside,openany]{book}
\usepackage{amd}
\usepackage{fancyvrb}
\usepackage{adjustbox}

% %--------------------------------------------------------------------------
% %         General Setting
% %--------------------------------------------------------------------------

\graphicspath{{Images/}{../Images/}} %Path of figures
\setkeys{Gin}{width=0.85\textwidth} %Size of figures
\setlength{\cftbeforechapskip}{3pt} %space between items in toc
\setlength{\parindent}{0.5cm} % Idk
% Theorem System
% The following boxes are provided:
%   Definition:     \defn 
%   Theorem:        \thm 
%   Lemma:          \lem
%   Corollary:      \cor
%   Proposition:    \prop   
%   Claim:          \clm
%   Fact:           \fact
%   Proof:          \pf
%   Example:        \ex
%   Remark:         \rmk (sentence), \rmkb (block)
% Suffix
%   r:              Allow Theorem/Definition to be referenced, e.g. thmr
%   p:              Add a short proof block for Lemma, Corollary, Proposition or Claim, e.g. lemp
%                   For theorems, use \pf for proof blocks

% ======= Real examples : 

% \defn{Definition Name}{
%     A defintion.
% }

% \thmr{Theorem Name}{mybigthm}{
%     A theorem.
% }

% \lem{Lemma Name}{
%     A lemma.
% }

% \fact{
%     A fact.
% }

% \cor{
%     A corollary.
% }

% \prop{
%     A proposition.
% }

% \clmp{}{
%     A claim.
% }{
%     A reference to Theorem~\ref{thm:mybigthm}
% }

% \pf{
%    proof.

% \ex{
%     some examples. 
% }

% \rmk{
%     Some remark.
% }

% \rmkb{
%     Some more remark.
% }

\usepackage{xcolor}

% Define custom colors
\definecolor{defscol}{HTML}{ecd8d7} %For definitions
\definecolor{asumscol}{HTML}{ecd8d7} %For Assumptions

\definecolor{rmkscol}{HTML}{313160} %For remarks
\definecolor{exmscol}{HTML}{e04b52} %For examples

\definecolor{lemscol}{HTML}{2c3943} %For Lemmes
\definecolor{thmscol}{HTML}{595765} %For Theorems
\definecolor{prpscol}{HTML}{9c98b1} %For proposition
\definecolor{corscol}{HTML}{dfd9fd} %For corrolaries

\definecolor{clmscol}{HTML}{165c58} %For claims
\definecolor{facscol}{HTML}{28a8a1} %For facts



% ============================
% Definition
% ============================
\newtcbtheorem[number within=section]{mydefinition}{Definition}
{
    enhanced,
    frame hidden,
    titlerule=0mm,
    toptitle=1mm,
    bottomtitle=1mm,
    fonttitle=\bfseries\large,
    coltitle=black,
    colbacktitle=defscol!40!white,
    colback=defscol!20!white,
}{defn}

\NewDocumentCommand{\defn}{m+m}{
    \begin{mydefinition}{#1}{}
        #2
    \end{mydefinition}
}

\NewDocumentCommand{\defnr}{mm+m}{
    \begin{mydefinition}{#1}{#2}
        #3
    \end{mydefinition}
}

% ============================
% Assumption
% ============================
\newtcbtheorem[use counter from=mydefinition]{myassumption}{Assumption}
{
    enhanced,
    frame hidden,
    titlerule=0mm,
    toptitle=1mm,
    bottomtitle=1mm,
    fonttitle=\bfseries\large,
    coltitle=black,
    colbacktitle=asumscol!40!white,
    colback=asumscol!20!white,
}{asum}

\NewDocumentCommand{\asum}{m+m}{
    \begin{myassumption}{#1}{}
        #2
    \end{myassumption}
}

\NewDocumentCommand{\asumr}{mm+m}{
    \begin{myassumption}{#1}{#2}
        #3
    \end{myassumption}
}

% ============================
% Theorem
% ============================

\newtcbtheorem[use counter from=mydefinition]{mytheorem}{Theorem}
{
    enhanced,
    frame hidden,
    titlerule=0mm,
    toptitle=1mm,
    bottomtitle=1mm,
    fonttitle=\bfseries\large,
    coltitle=black,
    colbacktitle=thmscol!40!white,
    colback=thmscol!20!white,
}{thm}

\NewDocumentCommand{\thm}{m+m}{
    \begin{mytheorem}{#1}{}
        #2
    \end{mytheorem}
}

\NewDocumentCommand{\thmr}{mm+m}{
    \begin{mytheorem}{#1}{#2}
        #3
    \end{mytheorem}
}

\newenvironment{thmpf}{
	{\noindent{\it \textbf{Proof for Theorem.}}}
	\tcolorbox[blanker,breakable,left=5mm,parbox=false,
    before upper={\parindent15pt},
    after skip=10pt,
	borderline west={1mm}{0pt}{thmscol!40!white}]
}{
    \textcolor{thmscol!40!white}{\hbox{}\nobreak\hfill$\blacksquare$} 
    \endtcolorbox
}

\NewDocumentCommand{\thmp}{m+m+m}{
    \begin{mytheorem}{#1}{}
        #2
    \end{mytheorem}

    \begin{thmpf}
        #3
    \end{thmpf}
}

% ============================
% Lemma
% ============================

\newtcbtheorem[use counter from=mydefinition]{mylemma}{Lemma}
{
    enhanced,
    frame hidden,
    titlerule=0mm,
    toptitle=1mm,
    bottomtitle=1mm,
    fonttitle=\bfseries\large,
    coltitle=black,
    colbacktitle=lemscol!40!white,
    colback=lemscol!20!white,
}{lem}

\NewDocumentCommand{\lem}{m+m}{
    \begin{mylemma}{#1}{}
        #2
    \end{mylemma}
}

\newenvironment{lempf}{
	{\noindent{\it \textbf{Proof for Lemma}}}
	\tcolorbox[blanker,breakable,left=5mm,parbox=false,
    before upper={\parindent15pt},
    after skip=10pt,
	borderline west={1mm}{0pt}{lemscol!40!white}]
}{
    \textcolor{lemscol!40!white}{\hbox{}\nobreak\hfill$\blacksquare$} 
    \endtcolorbox
}

\NewDocumentCommand{\lemp}{m+m+m}{
    \begin{mylemma}{#1}{}
        #2
    \end{mylemma}

    \begin{lempf}
        #3
    \end{lempf}
}

% ============================
% Corollary
% ============================

\newtcbtheorem[use counter from=mydefinition]{mycorollary}{Corollary}
{
    enhanced,
    frame hidden,
    titlerule=0mm,
    toptitle=1mm,
    bottomtitle=1mm,
    fonttitle=\bfseries\large,
    coltitle=black,
    colbacktitle=corscol!40!white,
    colback=corscol!20!white,
}{cor}

\NewDocumentCommand{\cor}{+m}{
    \begin{mycorollary}{}{}
        #1
    \end{mycorollary}
}

\newenvironment{corpf}{
	{\noindent{\it \textbf{Proof for Corollary.}}}
	\tcolorbox[blanker,breakable,left=5mm,parbox=false,
    before upper={\parindent15pt},
    after skip=10pt,
	borderline west={1mm}{0pt}{corscol!40!white}]
}{
    \textcolor{corscol!40!white}{\hbox{}\nobreak\hfill$\blacksquare$} 
    \endtcolorbox
}

\NewDocumentCommand{\corp}{m+m+m}{
    \begin{mycorollary}{}{}
        #1
    \end{mycorollary}

    \begin{corpf}
        #2
    \end{corpf}
}

% ============================
% Proposition
% ============================

\newtcbtheorem[use counter from=mydefinition]{myproposition}{Proposition}
{
    enhanced,
    frame hidden,
    titlerule=0mm,
    toptitle=1mm,
    bottomtitle=1mm,
    fonttitle=\bfseries\large,
    coltitle=black,
    colbacktitle=prpscol!30!white,
    colback=prpscol!20!white,
}{prop}

\NewDocumentCommand{\prop}{+m}{
    \begin{myproposition}{}{}
        #1
    \end{myproposition}
}

\newenvironment{proppf}{
	{\noindent{\it \textbf{Proof for Proposition.}}}
	\tcolorbox[blanker,breakable,left=5mm,parbox=false,
    before upper={\parindent15pt},
    after skip=10pt,
	borderline west={1mm}{0pt}{prpscol!40!white}]
}{
    \textcolor{prpscol!40!white}{\hbox{}\nobreak\hfill$\blacksquare$} 
    \endtcolorbox
}



\NewDocumentCommand{\propp}{+m+m}{
    \begin{myproposition}{}{}
        #1
    \end{myproposition}

    \begin{proppf}
        #2
    \end{proppf}
}

% ============================
% Claim
% ============================

\newtcbtheorem[use counter from=mydefinition]{myclaim}{Claim}
{
    enhanced,
    frame hidden,
    titlerule=0mm,
    toptitle=1mm,
    bottomtitle=1mm,
    fonttitle=\bfseries\large,
    coltitle=black,
    colbacktitle=clmscol!40!white,
    colback=clmscol!20!white,
}{clm}

\NewDocumentCommand{\clm}{m+m}{
    \begin{myclaim*}{#1}{}
        #2
    \end{myclaim*}
}

\newenvironment{clmpf}{
	{\noindent{\it \textbf{Proof for Claim.}}}
	\tcolorbox[blanker,breakable,left=5mm,parbox=false,
    before upper={\parindent15pt},
    after skip=10pt,
	borderline west={1mm}{0pt}{clmscol!40!white}]
}{
    \textcolor{clmscol!40!white}{\hbox{}\nobreak\hfill$\blacksquare$} 
    \endtcolorbox
}

\NewDocumentCommand{\clmp}{m+m+m}{
    \begin{myclaim*}{#1}{}
        #2
    \end{myclaim*}

    \begin{clmpf}
        #3
    \end{clmpf}
}

% ============================
% Fact
% ============================

\newtcbtheorem[use counter from=mydefinition]{myfact}{Fact}
{
    enhanced,
    frame hidden,
    titlerule=0mm,
    toptitle=1mm,
    bottomtitle=1mm,
    fonttitle=\bfseries\large,
    coltitle=black,
    colbacktitle=facscol!40!white,
    colback=facscol!20!white,
}{fact}

\NewDocumentCommand{\fact}{+m}{
    \begin{myfact}{}{}
        #1
    \end{myfact}
}

% ============================
% Proof
% ============================

\NewDocumentCommand{\pf}{+m}{
    \begin{proof}
        [\noindent\textbf{Proof.}]
        #1
    \end{proof}
}

% ============================
% Example
% ============================


\newenvironment{myexample}{
    \tcolorbox[blanker,breakable,left=5mm,parbox=false,
    before upper={\parindent15pt},
    after skip=10pt,
	borderline west={1mm}{0pt}{clmscol!40!white}]
}{
    \textcolor{clmscol!40!white}{\hbox{}\nobreak\hfill$\blacksquare$} 
    \endtcolorbox
}

\NewDocumentCommand{\exm}{m+m}{
    \begin{myexample}
	{\noindent{\it \textbf{Example : #1 }}}\\ 
        #2
    \end{myexample}
}


% ============================
% Remark
% ============================


\NewDocumentCommand{\rmk}{+m}{
    {\it \color{rmkscol!40!white}#1}
}

\newenvironment{remark}{
    \par
    \vspace{5pt}
    \begin{minipage}{\textwidth}
        {\par\noindent{\textbf{Remark.}}}
        \tcolorbox[blanker,breakable,left=5mm,
        before skip=10pt,after skip=10pt,
        borderline west={1mm}{0pt}{rmkscol!20!white}]
}{
        \endtcolorbox
    \end{minipage}
    \vspace{5pt}
}

\NewDocumentCommand{\rmkb}{+m}{
    \begin{remark}
        #1
    \end{remark}
}













% % Old styles hh 

% %--------------------------------------------------------------------------
% % 		THEOREMES STYLE
% %--------------------------------------------------------------------------

% %-------		DEFINITION 		-------	
% \newcounter{defo}[chapter]
% \newenvironment{defi}[1]{\refstepcounter{defo} 
% \begin{tcolorbox}[colback=yellow!20!white,colframe=yellow!15!black,title= \textbf{Définition \thechapter \ $\blacklozenge$ \thedefo \ | #1}]}{\end{tcolorbox}}

% %-------		THEOREME 		-------	
% \newcounter{th}[chapter]
% \newenvironment{thm}[1]{\refstepcounter{th}
% \begin{tcolorbox}[colback=mycolor!10,colframe=mycolor!10!black!80,title=\textbf{ Théorème \thechapter \ $\blacklozenge$ \theth \ | #1}]}{\end{tcolorbox}}

% %-------		PROPOSITION 		-------	
% \newcounter{prop}[chapter]
% \newenvironment{propt}[1]{\refstepcounter{prop}
% \begin{tcolorbox}[colback=mycolor!5,colframe=mycolor!10!linkscolor!80 ,title=\textbf{ Proposition \thechapter \ $\blacklozenge$ \theprop \ | #1}]}{\end{tcolorbox}}

% %-------		COROLLAIRE		-------
% \newcounter{cor}[chapter]
% \newenvironment{corr}[1]{\refstepcounter{cor}
% \begin{tcolorbox}[colback=mycolor!2,colframe=mycolor!10!linkscolor!40 ,title= Corolaire \thechapter \ $\blacklozenge$ \thecor \ | #1]}{\end{tcolorbox}}

% %-------		LEMME			-------
% \newcounter{lem}[chapter]
% \newenvironment{lemme}[1]{\refstepcounter{lem}
% \begin{tcolorbox}[colback=mycolor!10!blue!2,colframe=mycolor!50!blue!30,title=\textbf{Lemme \thechapter \ $\blacklozenge$ \thelem \ | #1}]}{\end{tcolorbox}}

% %-------		METHODE 			-------
% \newcounter{met}[chapter]
% \newenvironment{meth}[1]{\refstepcounter{met}
% \begin{tcolorbox}
% [enhanced jigsaw,breakable,pad at break*=1mm,
%  colback=red!20!white,boxrule=0pt,frame hidden,
%  borderline west={1.5mm}{-2mm}{red}] \color{red}
% {\textbf{Méthode \thechapter \ $\blacklozenge$ \themet \ | #1} } \color{black} \\ } {\end{tcolorbox}}

% %-------		REMARQUE 			-------
% \newcommand{\NB}[1]{
% \ \\
% \begin{tabular}{p{0.05\textwidth}p{0.80\textwidth}}
% \hline
% \vspace{-0.1cm} \includegraphics[scale=0.03]{./system/IDEA.png} & 	#1\\
% \hline
% \end{tabular}
% \ 
% \newline \ \newline
%  }

% %-------		EXEMPLE  		-------
% \newenvironment{exm}{ \begin{tcolorbox}
% [enhanced jigsaw,breakable,pad at break*=1mm,
%  colback=cyan!20!white,boxrule=0pt,frame hidden,
%  borderline west={1.5mm}{-2mm}{cyan}] \color{cyan}
% {\textbf{Exemple} } \color{black} \\ } {\end{tcolorbox}}
  % Theorems styles and colors
\usepackage[english]{babel} %Language
\usepackage{framed}

\setlist[itemize]{itemsep=3pt} % Adjust the length as needed
\setlist[enumerate]{itemsep=3pt} % Adjust the length as needed

\usepackage{lmodern} %  Latin Modern font

% %--------------------------------------------------------------------------
% %         General Informations
% %--------------------------------------------------------------------------
\newcommand{\BigTitle}{
    Compilers: Principles, Techniques, \& Tools
    }

\newcommand{\LittleTitle}{
    By Alfred V. Aho et all
    }

    
\begin{document}

% %--------------------------------------------------------------------------
% %         First pages 
% %--------------------------------------------------------------------------
\newgeometry{top=8cm,bottom=.5in,left=2cm,right=2cm}
\subfile{files/0.0.0.titlepage}
\restoregeometry
\thispagestyle{empty}
\setcounter{page}{0}
\tableofcontents
\thispagestyle{empty}
\setcounter{page}{0}

% %--------------------------------------------------------------------------
% %         Core of the document 
% %--------------------------------------------------------------------------
\chapter{Introduction}

The world as we know it depends on programming languages, because all the software running on all the computers was written in some programming language. But, before a program can be run, it first must be translated into a form in which it can be executed by a computer.

The software systems that do this translation are called \textit{compilers}.

\section{Language Processors}

Simply stated, a compiler is a program that can read a program in one language--the \textit{source} language--and translate it into an equivalent program in another language--the \textit{target} language.

An \textit{interpreter} is another common kind of language processor.

The task of collecting the source program is sometimes entrusted to a separate program, called a \textit{preprocessor}.

The compiler may produce an assembly-language program as its output, because assembly language is easier to produce as output and is easier to debug. The assembly language is then processed by a program called an \textit{assembler} that produces relocatable machine code as its output.

The \textit{linker} resolves external memory addresses, where the code in one file may refer to a location in another file. The \textit{loader} then puts together all of the executable object files into memory for execution.

\section{The Structure of a Compiler}

Up to this point we have treated a compiler as a single box that maps a source program into a semantically equivalent target program. If we open up this box a little, we see that there are two parts to this mapping: analysis and synthesis.

The \textit{analysis} part breaks up the source program into constituent pieces and imposes a grammatical structure on them. The analysis part also collects information about the source program and stores it in a data structure called a \textit{symbol table}, which is passed along with the intermediate representation to the synthesis part.

The \textit{synthesis} part constructs the desired target program from the intermediate representation and the information in the symbol table. The analysis part is often called the \textit{front end} of the compiler; the synthesis part is the \textit{back end}.

If we examine the compilation process in more detail, we see that it operates as a sequence of \textit{phases}, each of which transforms one representation of the source program to another.

\subsection{Lexical Analysis}

The first phase of a compiler is called \textit{lexical analysis} or \textit{scanning}. The lexical analyzer reads the stream of characters making up the source program and groups the characters into meaningful sequences called \textit{lexemes}. For each lexeme, the lexical analyzer produces as output a \textit{token} of the form $$\langle\textit{token-name, attribute-value}\rangle$$ that is passes on to the subsequent phase, syntax analysis. In the token, the first component \textit{token-name} is an abstract symbol that is used during syntax analysis, and the second component \textit{attribute-value} points to an entry in the symbol table for this token.

\subsection{Syntax Analysis}

The second phase of the compiler is \textit{syntax analysis} or \textit{parsing}. A typical representation is a \textit{syntax tree} in which each interior node represents an operation and the children of the node represent the arguments of the operation.

\subsection{Semantic Analysis}

The \textit{semantic analyzer} uses the syntax tree and the information in the symbol table to check the source program for semantic consistency with the language definition.

An important part of semantic analysis is \textit{type checking}, where the compiler checks that each operator has matching operands.

The language specification may permit some type conversions called \textit{coercions}.

\subsection{Intermediate Code Generation}

We consider an intermediate form called \textit{three-address code}, which consists of a sequence of assembly-like instructions with three operands per instruction.

\subsection{The Grouping of Phases into Passes}

In an implementation, activities from several phases may be grouped together into a \textit{pass} that reads an input file and writes an output file.

\subsection{Compiler-Construction Tools}

Some commonly used compiler-construction tools include
\begin{enumerate}
    \item \textit{Parser generators} that automatically produce syntax analyzers from a grammatical description of a programming language.
    \item \textit{Scanner generators} that produce lexical analyzers from a regular-expression description of the tokens of a language.
    \item \textit{Syntax-directed translation engines} that produce collections of routines for walking a parse tree and generating intermediate code.
    \item \textit{Code-generator generators} that produce a code generator from a collection of rules for translating each operation of the intermediate language into the machine language for a target machine.
    \item \textit{Data-flow analysis engines} that facilitate the gathering of information about how values are transmitted from one part of a program to each other part.
    \item \textit{Compiler-construction toolkits} that provide an integrated set of routines for construction various phases of a compiler.
\end{enumerate}

\section{The Evolution of Programming Language}
\subsection{The Move to Higher-Level Languages}

One classification is by generation. \textit{First-generation languages} are the machine languages, \textit{second-generation} the assembly languages, and \textit{third-generation} the higher-level languages. \textit{Fourth-generation languages} are languages designed for specific applications. The term \textit{fifth-generation language} has been applied to logic- and constraint-based languages.

Another classification of languages uses the term \textit{imperative} for languages in which a program specifies \textit{how} a computation is to be done and \textit{declarative} for languages in which a program specifies \textit{what} computation is to be done.

The term \textit{von Neumann language} is applied to programming languages whose computational model is based on the von Neumann computer architecture.

An \textit{object-oriented language} is one that supports object-oriented programming, a programming style in which a program consists of a collection of objects that interact with one another.

\textit{Scripting languages} are interpreted languages with high-level operators designed for "gluing together" computations.

\section{Applications of Compiler Technology}
\subsection{implementation of High-Level Programming Languages}

A body of compiler optimizations, known as \textit{data-flow optimizations}, has been developed to analyze the flow of data through the program and removes redundancies across these constructs.

Object-oriented programs are different from those written in many other languages, in that they consist of many more, but smaller, procedures (called \textit{methods} in object-oriented terms).

\subsection{Optimizations for Computer Architectures}

Almost all high-performance systems take advantage of the same two basic techniques: \textit{parallelism} and \textit{memory hierarchies}. Parallelism can be found at several levels: at the \textit{instruction level}, where multiple operations are executed simultaneously and at the \textit{processor level}, where different threads of the same application are run on different processors.

\section{Programming Language Basics}
\subsection{The Static/Dynamic Distinction}

If a language uses a policy that allows the compiler to decide an issue, then we say that the language uses a \textit{static} policy or that the issue can be decided at \textit{compile time}. On the other hand, a policy that only allows a decision to be made when we execute the program is said to be a \textit{dynamic policy} or to require a decision at \textit{run time}.

The \textit{scope} of a declaration of $x$ is the region of the program in which uses of $x$ refer to this declaration. A language uses \textit{static scope} or \textit{lexical scope} if it is possible to determine the scope of a declaration by looking only at the program. Otherwise, the language uses \textit{dynamic scope}.

\subsection{Environments and States}

\begin{figure}[htbp]
    \centering
    \begin{tikzpicture}
        \node (node1) at (0,0) [anchor=north] {names};
        \node (node2) at (3.5,0) [align=center, anchor=north] {locations\\(variables)};
        \node (node3) at (7,0) [anchor=north] {values};
        \draw [-latex, bend left=45] (node1.north east) to node[midway, above] {\textit{environment}} (node2.north west);
        \draw [-latex, bend left=45] (node2.north east) to node[midway, above] {\textit{state}} (node3.north west);
    \end{tikzpicture}
    \caption{Two-stage mapping from names to values}
    \label{figure:1.8}
\end{figure}

The association of names with locations in memory (the \textit{store}) and then with values can be described by two mappings that change as the program runs:
\begin{enumerate}
    \item The \textit{environment} is a mapping from names to locations in the store.
    \item The \textit{state} is a mapping from locations in store to their values.
\end{enumerate}

The environment and state mappings in Fig.\;\ref{figure:1.8} are dynamic, but there are a few exceptions:
\begin{enumerate}
    \item \textit{Static versus dynamic binding} of names to locations.
    \item \textit{Static versus dynamic binding} of locations to values.
\end{enumerate}

\begin{framed}
\begin{center}
    \textbf{{\large Names, Identifiers, and Variables}}
\end{center}

An \textit{identifier} is a string of characters, typically letters or digits, that refers to (identifies) an entity. Composite names are called \textit{qualified} names.

A \textit{variable} refers to a particular location of the store.
\end{framed}

\subsection{Static Scope and Block Structure}

The scope rules for C are based on program structure; the scope of a declaration is determined implicitly by where the declaration appears in the program. Later languages also provide explicit control over scopes through the use of keywords like \textbf{public}, \textbf{private} and \textbf{protected}.

A \textit{block} is a grouping of declarations and statements. C uses braces \verb|{| and \verb|}| to delimit a block; the alternative use of \textbf{begin} and \textbf{end} for the same purpose dates back to Algol.

In C, the syntax of blocks is given by
\begin{enumerate}
    \item One type of statement is a block. Blocks can appear anywhere that other types of statement can appear.
    \item A block is a sequence of declarations followed by a sequence of statements, all surrounded by braces.
\end{enumerate}

Note that this syntax allows blocks to be nested inside each other. This nesting property is referred to as \textit{block structure}.

\subsection{Explicit Access Control}

Through the use of keywords like \textbf{public}, \textbf{private}, and \textbf{protected}, object-oriented languages provide explicit control over access to member names in a superclass. These keywords support \textit{encapsulation} by restricting access.

\subsection{Dynamic Scope}

Technically, any scoping policy is dynamic if it is based on factor(s) that can be known only when the program executes. The term \textit{dynamic scope}, however, usually refers to the following policy: a use of a name $x$ refers to the declaration of $x$ in the most recently called procedure with such a declaration.

\begin{framed}
    \begin{center}
        \textbf{{\large Declarations and Definitions}}
    \end{center}

    In C++, a method is declared in a class definition, by giving the types of the arguments and result of the method (often called the \textit{signature} for the method).
\end{framed}

\subsection{Parameter Passing Mechanisms}

\textit{Actual parameters} (the parameters used in the call of a procedure) are associated with the \textit{formal parameters} (those used in the procedure definition).

\subsubsection{Call-by-Value}

In \textit{call-by-value}, the actual parameter is evaluated (if it is an expression) or copied (if it is a variable).

\subsubsection{Call-by-Reference}

In \textit{call-by-reference}, the address of the actual parameter is passed to the callee as the value of the corresponding formal parameter.

\subsection{Aliasing}

It is possible that two formal parameters can refer to the same location; such variables are said to be \textit{aliases} of one another.

\chapter{A Simple Syntax-Directed Translator}
\section{Introduction} 

The \textit{syntax} of a programming language defines what its programs, while the \textit{semantics} of the language defines what its program mean; that is, what each program does when it executes.

A lexical analyzer allows a translator to handle mutlicharacter constructs like identifiers, which are written as sequences of charactersm, but are treated as units called \textit{tokens} during syntax analysis.

\begin{figure}[htbp]
    \centering
    \begin{minipage}{0.45\linewidth}
        \centering
        \begin{tikzpicture}
            \node (node1) at (0,0) {\textbf{do-while}};
            \node (node2) at (-1.5,-1) {\textbf{body}};
            \node (node3) at (1.5,-1) {\verb|>|};
            \node (node4) at (-1.5,-2) {\textbf{assign}};
            \node (node5) at (1,-2) {\verb|[]|};
            \node (node6) at (2,-2) {$v$};
            \node (node7) at (-2,-3) {$i$};
            \node (node8) at (-1,-3) {\verb|+|};
            \node (node9) at (0.5,-3) {$a$};
            \node (node10) at (1.5,-3) {$i$};
            \node (node11) at (-1.5,-4) {$i$};
            \node (node12) at (-0.5,-4) {1};
            \draw ($(node1.south west)!0.25!(node1.south east)$) to ($(node2.north east)!0.25!(node2.north west)$);
            \draw ($(node1.south east)!0.25!(node1.south west)$) to (node3.north west);
            \draw (node2.south) to (node4.north);
            \draw ($(node3.south west)!0.25!(node3.south east)$) to ($(node5.north east)!0.25!(node5.north west)$);
            \draw ($(node3.south east)!0.25!(node3.south west)$) to ($(node6.north west)!0.25!(node6.north east)$);
            \draw ($(node4.south west)!0.4!(node4.south east)$) to ($(node7.north east)!0.25!(node7.north west)$);
            \draw ($(node4.south east)!0.4!(node4.south west)$) to ($(node8.north west)!0.25!(node8.north east)$);
            \draw ($(node5.south west)!0.25!(node5.south east)$) to ($(node9.north east)!0.25!(node9.north west)$);
            \draw ($(node5.south east)!0.25!(node5.south west)$) to ($(node10.north west)!0.25!(node10.north east)$);
            \draw ($(node8.south west)!0.25!(node8.south east)$) to ($(node11.north east)!0.25!(node11.north west)$);
            \draw ($(node8.south east)!0.25!(node8.south west)$) to ($(node12.north west)!0.25!(node12.north east)$);
        \end{tikzpicture}
        \subcaption{}
    \end{minipage}
    \begin{minipage}{0.45\linewidth}
        \centering
        \begin{tabular}{l}
             \verb|1:|\quad\verb|i = i + 1|\\
             \verb|2:|\quad\verb|t1 = a [ i ]|\\
             \verb|3:|\quad\verb|if t1 < v goto 1|
        \end{tabular}
        \subcaption{}
    \end{minipage}
    \caption{Intermediate code for "\texttt{do i=i+1; while(a[i]<v);}"}
    \label{figure:2.4}
\end{figure}

Two forms of intermediate code are illustrated in Fig.\;\ref{figure:2.4}. One form, called \textit{abstract syntax trees} or simply \textit{syntax trees}, represents the hierarchical systematic structure of the source program.

\section{Syntax Definition}

A grammar naturally describes the hierarchical structure of most programming language constructs. For example, an if-else statement in Java can have the form
\begin{center}
    \textbf{if} (expression) statement \textbf{else} statement
\end{center}

Using the variable \textit{expr} to denote an expression and the variable \textit{stmt} to denote a statement, this structuring rule can be expressed as
\begin{center}
    \textit{stmt} $\rightarrow$ \textbf{if} (\textit{expr}) \textit{stmt} \textbf{else} \textit{stmt}
\end{center}
in which the arrow may be read as "can have the form." Such a rule is called a \textit{production}. In a production, lexical elements are called \textit{terminals}. Variables like \textit{expr} and \textit{stmt} represent sequences of terminals and are called \textit{nonterminals}.

\subsection{Definition of Grammars}

A \textit{context-free grammar} has four components:
\begin{enumerate}
    \item A set of \textit{terminal} symbols, sometimes referred to as "tokens."
    \item A set of \textit{nonterminals}, sometimes called "syntactic variables."
    \item A set of \textit{productions}, where each production consists of a nonterminal, called the \textit{head} or \textit{left side} of the production, an arrow, and a sequence of terminals and/or nonterminals, called the \textit{body} or \textit{right side} of the production.
    \item A designation of one of the nonterminals as the \textit{start} symbol.
\end{enumerate}

\begin{framed}
    \begin{center}
        \textbf{{\large Tokens Versus Terminals}}
    \end{center}

    A token consists of two components, a token name and an attribute value. The token names are abstract symbols that are used by the parser for syntax analysis. Often, we shall call these token names \textit{terminals}, since they appear as terminal symbols in the grammar for a programming language.
\end{framed}

We say a production is \textit{for} a nonterminal if the nonterminal is the head of the production. The string of zero terminals, written as $\epsilon$, is called the \textit{empty} string.

\subsection{Derivations}

The terminal strings that can be derived from the start symbol form the \textit{language} defined by the grammar.

\textit{Parsing} is the problem of taking a string of terminals and figuring out how to derive it from the start symbol of the grammar, and if it cannot be derived from the start symbol of the grammar, then reporting syntax errors within the string.

\subsection{Parse Trees}

A parse tree pictorially shows how the start symbol of a grammar derives a string in the language.

Formally, given a context-free grammar, a \textit{parse tree} according to the grammar is a tree with the following properties:
\begin{enumerate}
    \item The root is labeled by the start symbol.
    \item Each leaf is labeled by a terminal or by $\epsilon$.
    \item Each interior node is labeled by a nonterminal.
    \item If $A$ is the nonterminal labeling some interior node and $X_1,X_2,\cdots,X_n$ are the labels of the children of that node from left to right, then there must be a production $A\rightarrow X_1X_2\cdots X_n$.
\end{enumerate}

\begin{framed}
    \begin{center}
        \textbf{{\large Tree Terminology}}
    \end{center}

    Tree data structures figure prominently in compiling.
    \begin{itemize}
        \item A tree consists of one or more \textit{nodes}. Nodes may have \textit{labels}.
        \item Exactly one node is the \textit{root}. All nodes except the root have a unique \textit{parent}; the root has no parent.
        \item If node $N$ is the parent of node $M$, then $M$ is a \textit{child} of $N$. The children of one node are called \textit{siblings}. They have an order, \textit{from the left}, and when we draw trees, we order the children 
        \item A node with no children is called a \textit{leaf}. Other nodes -- those with one or more children -- are \textit{interior nodes}.
        \item A \textit{descendant} of a node $N$ is either $N$ itself, a child of $N$, a child of a child of $N$, and so on, for any number of levels. We say node $N$ is an \textit{ancestor} of node $M$ if $M$ is a descendant of $N$.
    \end{itemize}
\end{framed}

From left to right, the leaves of a parse tree form the \textit{yield} of the tree, which is the string \textit{generated} or \textit{derived} from the nonterminal at the root of the parse tree.

The process of finding a parse tree for a given string of terminals is called \textit{parsing} that string.

\subsection{Ambiguity}

We have to be careful in talking about \textit{the} structure of a string according to a grammar. A grammar can have more than one parse tree generating a given string of terminals. Such a grammar is said to be \textit{ambiguous}.

\subsection{Associativity of Operators}

We say that the operator $+$ \textit{associates} to the left, because an operand with plus signs on both sides of it belongs to the operator to its left.

\subsection{Precedence of Operators}

We say that $*$ has \textit{higher precedence} than $+$ if $*$ takes its operands before $+$ does.

\section{Syntax-Directed Translation}

This section introduces two concepts related to syntax-directed translation:
\begin{itemize}
    \item\textit{Attributes}. An \textit{attribute} is any quantity associated with a programming construct.
    \item(\textit{Syntax-directed}) \textit{translation schemes}. A \textit{translation scheme} is a notation for attaching program fragments to the productions of a grammar.
\end{itemize}

\subsection{Postfix Notation}

The \textit{postfix notation} for an expression $E$ can be defined inductively as follows:
\begin{enumerate}
    \item If $E$ is a variable or constant, then the postfix notation for $E$ is $E$ itself.
    \item If $E$ is an expression of the form $E_1$ \textbf{op} $E_2$, where \textbf{op} is any binary operator, then the postfix notation for $E$ is $E_1'$ $E_2'$ \textbf{op}, where $E_1'$ and $E_2'$ are the postfix notations for $E_1$ and $E_2$, respectively.
    \item If $E$ is a parenthesized expression of the form $(E_1)$, then the postfix notation for $E$ is the same as the postfix notation for $E_1$.
\end{enumerate}

No parentheses are needed in postfix notation, because the position and \textit{arity} (number of arguments) of the operators permits only one decoding of a postfix expression.

\subsection{Synthesized Attributes}

A \textit{syntax-directed definition} associates:
\begin{enumerate}
    \item With each grammar symbol, a set of attributes, and
    \item With each production, a set of \textit{semantic rules} for computing the values of the attributes associated with the symbols appearing in the production.
\end{enumerate}

A parse tree showing the attribute values at each node is called an \textit{annotated} parse tree.

An attribute is said to be \textit{synthesized} if its value at a parse-tree node $N$ is determined from attribute values at the children of $N$ and at $N$ itself.

\subsection{Tree Traversals}

A \textit{traversal} of a tree starts at the root and visits each node of the tree in some order.

A \textit{depth-first} traversal starts at the root and recursively visits the children of each node in any order, not necessarily from left to right.

Synthesized attributes can be evaluated during any \textit{bottom-up} traversal, that is, a traversal that evaluates attributes at a node after having evaluated attributes at its children.

\subsection{Translation Schemes}

\begin{framed}
    \begin{center}
        \textbf{{\large Preorder and Postorder Traversals}}
    \end{center}

    Often, we traverse a tree to perform some particular action at each node. If the action is done when we first visit a node, then we may refer to the traversal as a \textit{preorder traversal}. Similarly, if the action is done just before we leave a node for the last time, then we say it is a \textit{postorder traversal} of the tree.

    The \textit{preorder} of a (sub)tree rooted at node $N$ consists of $N$, followed by the preorders of the subtrees of each of its children, if any, from the left. The \textit{postorder} of a (sub)tree rooted at $N$ consists of the postorders of each of the subtrees for the children of $N$, if any, from the left, followed by $N$ itself.
\end{framed}

Program fragments embedded within production bodies are called \textit{semantic actions}.

\section{Parsing}

Most parsing methods fall into one of two classes, called the \textit{top-down} and \textit{bottom-up} methods.

\subsection{Top-Down Parsing}

The current terminal being scanned in the input is frequently referred to as the \textit{lookahead} symbol.

\subsection{Predictive Parsing}

\textit{Recursive-descent parsing} is a top-down method of syntax analysis in which a set of recursive procedures is used to process the input. Here, we consider a simple form of recursive-descent parsing, called \textit{predictive parsing}, in which the lookahead symbol unambiguously determines the flow of control through the procedure body for each nonterminal.

\subsection{Designing a Predictive Parser}

Recall that a \textit{predictive parser} is a program consisting of a procedure for every nonterminal.

\subsection{Left Recursion}

Consider a nonterminal $A$ with tow productions $$A\;\rightarrow\;A\alpha\;|\;\beta$$ where $\alpha$ and $\beta$ are sequences of terminals and nonterminals that do not start with $A$.

The nonterminal $A$ and its production are said to be \textit{left recursive}, because the production $A\to A\alpha$ has $A$ itself as the leftmost symbol on the right side.

The same effect can be achieved by rewriting the productions for $A$ in the following manner, using a new nonterminal $R$:
\begin{align*}
    &A\;\to\;\beta R\\
    &R\;\to\;\alpha R\;|\;\epsilon
\end{align*}

Nonterminal $R$ and its production $R\to\alpha R$ are \textit{right recursive} because this production for $R$ has $R$ itself as the last symbol on the right side.

\section{A Translator for Simple Expressions}

\subsection{Abstract and Concrete Syntax}

In an \textit{abstract syntax tree} for an expression, each interior node represents an operator; the children of the node represent the operands of the operator.

Abstract syntax trees, or simply \textit{syntax trees}, resemble parse trees to an extent. Many nonterminals of a grammar represent programming constructs, but others are "helpers" of one sort of another. In the syntax tree, these helpers typically are not needed and are hence droped. To emphasize the contrast, a parse tree is sometimes called a \textit{concrete syntax tree}, and the underlying grammar is called a \textit{concrete syntax} for the language.

\begin{figure}[htbp]
    \centering
    \begin{tikzpicture}
        \node (node1) at (0,0) {\verb|+|};
        \node (node2) at (-1,-1) {\verb|-|};
        \node (node3) at (1,-1) {\verb|2|};
        \node (node4) at (-2,-2) {\verb|9|};
        \node (node5) at (0,-2) {\verb|5|};
        \draw (node1.south west) to (node2.north east);
        \draw (node1.south east) to (node3.north west);
        \draw (node2.south west) to (node4.north east);
        \draw (node2.south east) to (node5.north west);
    \end{tikzpicture}
    \caption{Syntax tree for \texttt{9-5+2}}
    \label{Figure:2.22}
\end{figure}

In the syntax tree in Fig.\;\ref{Figure:2.22}, each interior node is associated with an operator, with no "helper" nodes for \textit{single productions} (a production whose body consists of a single nonterminal, and nothing else) or for $\epsilon$-productions.

\subsection{Simplifying the Translator}

When the last statement executed in a procedure body is a recursive call to the same procedure, the call is said to be \textit{tail recursive}.

\subsection{The Complete Program}

\begin{Verbatim}
    import java.io.*
    class Parser {
        static int lookahead;

        public Parser() throws IOException {
            lookahead = System.in.read();
        }

        void expr() throws IOException {
            term();
            while(true) {
                if (lookahead == '+') {
                    match('+'); term(); System.out.write('+');
                }
                else if (lookahead == '-') {
                    match('-'); term(); System.out.write('-');
                }
                else return;
            }
        }

        void term() throws IOException {
            if (Character.isDigit((char)lookahead)) {
                System.out.write((char)lookahead); match(lookahead);
            }
            else throw new Error("syntax error");
        }
    }

    public class Postfix {
        public static void main(String[] args) throws IOException {
            Parser parse = new Parser();
            parse.expr(); System.out.write('\n');
        }
    }
\end{Verbatim}
\begin{figure}[htbp]
    \caption{Java program to translate infix expressions into postfix form}
    \label{Figure:2.27}
\end{figure}

The function \verb|Parser|, with the same name as its class, is a \textit{constructor}; it is called automatically when an object of the class is created.

The construction \verb|(char)lookahead| \textit{casts} or coerces \verb|lookahead| to be a character.

\section{Lexical Analysis}

A sequence of input characters that comprises a single token is called a \textit{lexeme}.

\subsection{Recognizing Keywords and Identifiers}

Most languages use fixed character strings as punctuation marks or to identify consturcts. Such character strings are called \textit{keywords}.

Keywords generally satisfy the rules for forming identifiers, so a mechanism is needed f deciding when a lexeme forms a keyword and when it forms an identifier. The problem is easier to resolve if keywords are \textit{reserved}; i.e., if they cannot be used as identifiers.

The lexical analyzer in this section solves two problems by using a table to hold character strings:
\begin{itemize}
    \item\textit{Single Representation.}
    \item\textit{Reserved Words}.
\end{itemize}

In Java, a string table can be implemented as a hash table using a class called \textit{Hashtable}.

\section{Symbol Tables}

\textit{Symbol tables} are data structures that are used by compilers to hold information about source-program constructs.

\subsection{Symbol Table Per Scope}

The term \textit{scope} by itself refers to a portion of a program that is the scope of one or more declarations.

The \textit{most-closely nested} rule for blocks is that an identifier $x$ is in the scope of the most-closely nested declaration of $x$; that is, the declaration of $x$ found by examining blocks inside-out, starting with the block in which $x$ appears.

\section{Intermediate Code Generation}
\subsection{Two Kinds of Intermediate Representations}

In addition to creating an intermediate representation, a compiler front end checks that the source program follows the syntactic and semantic rules of the source language. This checking is called \textit{static checking}; in general "static" means "done by  the compiler."

\subsection{Static Checking}

Static checking includes:
\begin{itemize}
    \item\textit{Syntactic Checking}.
    \item\textit{Type Checking}.
\end{itemize}

\subsubsection{L-values and R-values}

The terms \textit{l-value} and \textit{r-value} refer to values that are appropriate on the left and right sides of an assignment, respectively.

\subsubsection{Type Checking}

Type checking assures that the type of a construct matches that expected by its context.

The idea of matching actual with expected types continues to apply, even in the following situations:
\begin{itemize}
    \item\textit{Coercions}. A \textit{coercion} occurs if the type of an operand is automatically converted to the type expected by the operator.
    \item\textit{Overloading}. A symbol is said to be \textit{overloaded} if it has different meanings depending on its context.
\end{itemize}

\chapter{Lexical Analysis}

To implement a lexical analyzer by hand, it helps to start with a diagram or other description for the lexemes of each token. We can then write code to identify each occurrence of each lexeme on the input and to return information about the token identified.

We can also produce a lexical analyzer automatically by specifying the lexeme patterns to a \textit{lexical-analysis generator} and compiling those patterns into code that functions as a lexical analyzer.

\section{The Role of the Lexical Analyzer}

Since the lexical analyzer is the part of the compiler that reads the source text, it may perform certain other tasks besides identification of lexemes. One such task is stripping out comments and \textit{whitespace} (blank, newline, tab, and perhaps other characters that are used to separate tokens in the input).

Sometimes, lexical analyzers are divided into a cascade of two processes:
\begin{itemize}
    \item[a)]\textit{Scanning} consists of the simple processes that do not require tokenization of the input.
    \item[b)]\textit{Lexical analysis} proper is the more complex portion, where the scanner produces the sequence of tokens as output.
\end{itemize}

\subsection{Tokens, Patterns, and Lexemes}

When discussing lexical analysis, we use three related but distinct terms:
\begin{itemize}
    \item A \textit{token} is a pair consisting of a token name and an optional attribute value.
    \item A \textit{pattern} is a description of the form that the lexemes of a token may take. In the case of a keyword as a token, the pattern is just the sequence of characters that form the keyword. For identifiers and some other tokens, the pattern is a more complex structure that is \textit{matched} by many strings.
    \item A \textit{lexeme}  is a sequence of characters in the source program that matches the pattern for a token and is identified by the lexical analyzer as an instance of that token.
\end{itemize}

\section{Specification of Tokens}

\begin{framed}
    \begin{center}
        \textbf{{\large Can We Run Out of Buffer Space}}
    \end{center}

    To avoid problems with long character strings, we can treat them as a concatenation of components, one from each line over which the string is written.

    A more difficult problem occurs when arbitrarily long lookahead may be needed. For example, some languages like PL/I do not treat keywords as \textit{reserved}; that is, you can use identifiers with the same name as a keyword.
\end{framed}

\subsection{Strings and Languages}

An \textit{alphabet} is any finite set of symbols. The set $\{0,1\}$ is the \textit{binary alphabet}.

A \textit{string} over an alphabet is a finite sequence of symbols drawn from that alphabet. The \textit{empty string}, denoted $\epsilon$, is the string of length zero.

A \textit{language} is any countable set of strings over some fixed alphabet. Abstract languages like $\emptyset$, the \textit{empty set} are languages under this definition.

\begin{framed}
    \begin{center}
        \textbf{{\large Terms for Parts of Strings}}
    \end{center}

    The following string-related terms are commonly used:
    \begin{enumerate}
        \item A \textit{prefix} of string $s$ is any string obtained by removing zero or more symbols from the end of $s$.
        \item A \textit{suffix} of string $s$ is any string obtained by removing zero or more symbols from the beginning of $s$.
        \item A \textit{substring} of $s$ is obtained by deleting any prefix and any suffix from $s$.
        \item The \textit{proper} prefixes, suffixes, and substrings of a string $s$ are those, prefixes, suffixes, and substrings, respectively, of $s$ that are not $\epsilon$ or not equal to $s$ itself.
        \item A \textit{subsequence} of $s$ is any string formed by deleting zero or more not necessarily consecutive positions of $s$.
    \end{enumerate}
\end{framed}

If $x$ and $y$ are strings, then the \textit{concatenation} of $x$ and $y$, denoted $xy$, is the string formed by appending $y$ to $x$.

\subsection{Operations and Languages}

The (\textit{Kleene}) \textit{closure} of a language $L$, denoted $L^*$, is the set of strings you get by concatenating $L$ zero or more times.

\subsection{Regular Expressions}

A language that can be defined by a regular expression is called a \textit{regular set}. If two regular expressions $r$ and $s$ denote the same regular set, we say they are \textit{equivalent} and write $r=s$.

\subsection{Regular Definitions}

If $\Sigma$ is an alphabet of basic symbols, then a \textit{regular definition} is a sequence of definitions of the form: $$\begin{array}{ccc}d_1&\to&r_1\\d_2&\to&r_2\\&\cdots\\d_n&\to&r_n\end{array}$$ where:
\begin{enumerate}
    \item Each $d_i$ is a new symbol, not in $\Sigma$ and not the same as any other of the $d$'s, and
    \item Each $r_i$ is a regular expression over the alphabet $\Sigma\cup\{d_1,d_2,\ldots,d_{i-1}\}$.
\end{enumerate}

\subsection{Extensions of Regular Expressions}

Here we mention a few notational extensions that were first incorporated into Unix utilities that are particularly useful in the specification lexical analyzer.
\begin{enumerate}
    \item\textit{One or more instances}.
    \item\textit{Zero or one instance}.
    \item\textit{Character classes}.
\end{enumerate}

\section{Recognition of Tokens}
\subsection{Transition Diagrams}

\textit{Transition diagrams} have a collection of nodes or circles, called \textit{states}.

\textit{Edges} are directed from one state of the transition diagram to another. Each edge is \textit{labeled} by a symbol or set of symbols. We shall assume that all our transition diagrams are \textit{deterministic}, meaning that there is never more than one edge out of a given state with a given symbol among its labels. Some important conventions about transition diagrams are:
\begin{itemize}
    \item Certain states are said to be \textit{accepting}, or \textit{final}.
    \item One state is designated the \textit{start state}, or \textit{initial state}; it is indicated by an edge, labeled "start," entering from nowhere.
\end{itemize}

\subsection{Recognition of Reserved Words and Identifiers}

Usually, keywords are reserved, so they are not identifiers even though they \verb|look| like identifiers.

\section{The Lexical-Analyzer Generator \texttt{Lex}}

The input notation for the \verb|Lex| tool is referred to as the \textit{Lex language} and the tool itself is the \textit{Lex compiler}.

\subsection{Structure of \texttt{Lex} Programs}

A \verb|Lex| program has the following form:
\begin{equation*}
    \begin{aligned}&\text{declarations}\\&\%\%\\&\text{transition rules}\\&\%\%\\&\text{auxiliary functions}\end{aligned}
\end{equation*}
The declarations section includes declarations of variables, \textit{manifest constants} (identifiers declared to stand for a constant).

\section{Finite Automata}
We shall now discover how \verb|Lex| turns its input program into a lexical analyzer. At the heart of the transition is the formalism known as \textit{finite automata}. These are essentially graphs, like transition diagrams, with a few differences:
\begin{enumerate}
    \item Finite automata are \textit{recognizers};  they simply say "yes" or "no" about each possible input string.
    \item Finite automata come in two flavors:
    \begin{itemize}
        \item[(a)]\textit{Nondeterministic finite automata} (NFA) have no restrictions on the labels of their edges.
        \item[(b)]\textit{Deterministic finite automata} (DFA) have, for each state, and for each symbol of its input alphabet exactly one edge with that symbol leaving that state.
    \end{itemize}
\end{enumerate}

Both deterministic and nondeterministic finite automata are capable of recognizing the same languages. In fact these languages are exactly the same languages, called the \textit{regular languages}, the regular expressions can describe.\footnote{There is a small lacuna: as we defined them, regular expressions cannot describe the empty language, since we would never want to use this pattern in practice. However, finite automata \textit{can} define the empty language.}

\subsection{Nondeterministic Finite Automata}

A \textit{nondeterministic finite automata} (NFA) consists of:
\begin{enumerate}
    \item A finite set of state $S$.
    \item A set of input symbols $\Sigma$, the \textit{input alphabet}.
    \item A \textit{transition function} that gives, for each state, and for each symbol in $\Sigma\cup\{\epsilon\}$ a set of \textit{next states}.
    \item A state $s_0$ from $S$ that is distinguished as the \textit{start state} (or \textit{initial state}).
    \item A set of states $F$, a subset of $S$, that is distinguished as the \textit{accepting states} (or \textit{final states}).
\end{enumerate}
 
We can represent either an NFA or DFA by a \textit{transition graph}, where the nodes are states and the labeled edges represent the transition function.

\subsection{Transition Tables}

We can also represent an NFA by a \textit{transition table}, whose rows correspond to states, and whose columns correspond to the input symbols and $\epsilon$.

\subsection{Acceptance of Input Strings by Automata}

An NFA \textit{accepts} input string $x$ if and only if there is some path in the transition graph from the start state to one of the accepting staets, such that the symbols along the path spell out $x$.

The \textit{language defined} (or \textit{accepted}) by an NFA is the set of strings labeling some path from the start to an accepting state.

\subsection{Deterministic Finite Automata}

A \textit{deterministic finite automata} (DFA) is a special case of an NFA where:
\begin{enumerate}
    \item There are no moves on input $\epsilon$, and
    \item For each state $s$ and input symbol $a$, there is exactly one edge out of $s$ labeled $a$.
\end{enumerate}

\exm{Simulating a DFA.}{
    \noindent{\small\textbf{INPUT:}} An input string $x$ terminated by an end-of-file character \textbf{eof}. A DFA $D$ with start state $s_0$, accepting states $F$, and transition function $move$.

    \noindent{\small\textbf{OUTPUT:}} Answer "yes" if $D$ accepts $x$; "no" otherwise.

    \noindent{\small\textbf{METHOD:}} Apply the algorithm in Fig.\;\ref{Figure:3.27} to the input string $x$. The function $move(s,c)$ gives the state to which there is an edge from state $s$ on input $c$. The function $nextChar$ returns the next character of the input string $x$.
}

\begin{center}
    \begin{tabular}{l}
        $s=s_0$;\\
        $c=nextChar()$;\\
        \textbf{while} ( $c$ !$=$ \textbf{eof} ) \{\\
        \qquad $s=move(s,c)$;\\
        \qquad $c=nextChar()$;\\
        \}\\
        \textbf{if} ( $s$ is in $F$ ) \textbf{return} \verb|"yes"|;\\
        \textbf{else return} \verb|"no"|;
    \end{tabular}
\end{center}
\begin{figure}[htbp]
    \caption{Simulating a DFA}
    \label{Figure:3.27}
\end{figure}

\section{From Regular Expressions to Automata}
\subsection{Conversion of an NFA to a DFA}

\exm{The \textit{subset construction} of a DFA from an NFA.}{
    \noindent{\small\textbf{INPUT:}} An NFA $N$.

    \noindent{\small\textbf{OUTPUT:}} A DFA $D$ accepting the same language as $N$.

    \noindent{\small\textbf{METHOD:}} Our algorithm constructs a transition table $Dtran$ for $D$. Each state of $D$ is a set of NFA states, and we construct $Dtran$ so $D$ will simulate "in parallel" all possible moves $N$ can make on a given input string. Our first problem is to deal with $\epsilon$-transitions of $N$ properly. In Fig.\;\ref{Figure:3.31} we see the definitions of several functions that describe basic computations on the states of $N$ that are needed in the algorithm. Note that $s$ is a single state of $N$, while $T$ is a set of states of $N$.

    We must explore those sets of states that $N$ can be in after seeing some input string. As a basis, before reading the first input symbol, $N$ can be in any of the states of $\epsilon$-$closure(s_0)$, where $s_0$ is its start state. For the induction, suppose that $N$ can be in set of states $T$ after reading input string $x$. If it next reads input $a$, then $N$ can immediately go to any of the states in $move(T,a)$. However, after reading $a$, it may also make several $\epsilon$-transitions; thus $N$ could be in any state of $\epsilon$-$closure(move(T,a))$ after reading input $xa$. Following these ideas, the construction of the set of $D$'s states, $Dstates$, and its transition function $Dtran$, is shown in Fig.\;\ref{Figure:3.32}.
    
    The start state of $D$ is $\epsilon$-$closure(s_0)$, and the accepting states of $D$ are all those sets of $N$'s states that include at least one accepting state of $N$. To complete our description of the subset construction, we need only to show how initially, $\epsilon$-$closure(s_0)$ is the only state in $Dstates$, and it is unmarked; $\epsilon$-$closure(T)$ is computed for any set of NFA states $T$. This process, shown in Fig.\;\ref{Figure:3.33}, is a straightforward search in graph from a set of states. In this case, imagine that only the $\epsilon$-labeled edges are available in the graph.
}

\begin{tabular}{p{0.15\columnwidth}|p{0.75\columnwidth}}
    \hline
    \multicolumn{1}{c|}{Operation} & \multicolumn{1}{c}{Description} \\ \hline
    $\epsilon$-$closure(s)$ & Set of NFA states reachable from NFA state $s$ on $\epsilon$-transitions alone. \\ \hline
    $\epsilon$-$closure(T)$ & Set of NFA states reachable from NFA state $s$ in set $T$ on $\epsilon$-transitions alone; $=\cup_{s\text{ in }T}\epsilon$-$closure(s)$. \\ \hline
    $move(T,a)$ & Set of NFA states to which there is transition on input symbol $a$ from some table $s$ in $T$. \\ \hline
\end{tabular}
\begin{figure}[htbp]
    \caption{Operations on NFA states}
    \label{Figure:3.31}
\end{figure}

\begin{center}
    \begin{tabular}{l}
        \textbf{while} ( there is an unmarked state $T$ in $Dstates$ ) \{\\
        \qquad mark $T$;\\
        \qquad\textbf{for} ( each input symbol $a$ ) \{\\
        \qquad\qquad$U=\epsilon$-$closure(move(T,a))$;\\
        \qquad\qquad\textbf{if} ( $U$ is not in $Dstates$ )\\
        \qquad\qquad\qquad add $U$ as an unmarked state to $Dstates$;\\
        \qquad\qquad$Dtran[T,a]=U$;\\
        \qquad\}\\
        \}
    \end{tabular}
\end{center}
\begin{figure}[htbp]
    \caption{The subset construction}
    \label{Figure:3.32}
\end{figure}

\begin{center}
    \begin{tabular}{l}
        push all states of $T$ onto $stack$;\\
        initialize $\epsilon$-$closure(T)$ to $T$;\\
        \textbf{while} ( $stack$ is not empty ) \{\\
        \qquad pop $t$, the top element, off $stack$;\\
        \qquad\textbf{for} ( each state $u$ with an edge from $t$ to $u$ labeled $\epsilon$ )\\
        \qquad\qquad\textbf{if} ( $u$ is not in $\epsilon$-$closure(T)$ ) \{\\
        \qquad\qquad\qquad add $u$ to $\epsilon$-$closure(T)$;\\
        \qquad\qquad\qquad push $u$ onto $stack$;\\
        \qquad\qquad\}\\
        \}
    \end{tabular}
\end{center}
\begin{figure}[htbp]
    \caption{Computing $\epsilon$-$closure(T)$}
    \label{Figure:3.33}
\end{figure}

\subsection{Simulation of an NFA}

\exm{Simulating an NFA.}{
    \noindent{\small\textbf{INPUT:}} An input string $x$ terminated by an end-of-file character \textbf{eof}. An NFA $N$ with start state $s_0$, accepting states $F$, and transition function $move$.

    \noindent{\small\textbf{OUTPUT:}} Answer "yes" if $M$ accepts $x$; "no" otherwise.

    \noindent{\small\textbf{METHOD:}} The algorithm keeps a set of current states $S$, those that are reached from $s_0$ following a path labeled by the inputs read so far. If $c$ is the next input character, read by function $nextChar()$, then we first compute $move(S,c)$ and then close that set using $\epsilon-closure()$. The algorithm is sketched in Fig\;\ref{Figure:3.37}.
}

\begin{center}
    \begin{tabular}{l}
        1)\quad$S=\epsilon$-$closure(s_0)$;\\
        2)\quad$c=nextChar()$;\\
        3)\quad\textbf{while} ( $c$ !$=$ \textbf{eof} ) \{\\
        4)\quad\qquad$S=\epsilon$-$closure(move(S,c))$;\\
        5)\quad\qquad$c=nextChar()$;\\
        6)\quad\}\\
        7)\quad\textbf{if} ( $S\cap F$ !$=\varnothing$ ) \textbf{return} \verb|"yes"|;\\
        8)\quad\textbf{else return} \verb|"no"|;
    \end{tabular}
\end{center}
\begin{figure}[htbp]
    \caption{Simulating an NFA}
    \label{Figure:3.37}
\end{figure}

\subsection{Efficiency of NFA Simulation}

\begin{framed}
    \begin{center}
        \textbf{{\large Big-Oh Notation}}
    \end{center}

    Technically, we say a function $f(n)$, perhaps the running time of some step of an algorithm, is $O(g(n))$  if there are constants $c$ and $n_0$, such that whenever $n\ge n_0$, it is true that $f(n)\le cg(n)$. The use of this \textit{big-oh notation} enables us to avoid getting too far into the details of what we count as a unit of execution time, yet lets us express the rate at which the running time of an algorithm grows.
\end{framed}

\subsection{Construction of an NFA from a Regular Expression}

\exm{The McNaughton-Yamada-Thompson algorithm to convert a regular expression to an NFA.}{
    \noindent{\small\textbf{INPUT:}} A regular expression $r$ over alphabet $\Sigma$.

    \noindent{\small\textbf{OUTPUT:}} An NFA $N$ accepting $L(r)$.

    \noindent{\small\textbf{METHOD:}} Begin by parsing $r$ into its constituent subexpressions. The rules for constructing an NFA consist of basic rules for handling subexpressions with no operators, and inductive rules for constructing larget NFA's from the NFA's for the immediate subexpressions of a given expression.

    \noindent{\small\textbf{BASIS:}} For expression $\epsilon$ construct the NFA
    \begin{center}
        \begin{tikzpicture}
            \node (node1) at (-0.5,0) {$i$};
            \node (node2) at (2,0) {$f$};
            \draw (-0.5,0) circle [radius=0.25];
            \draw (2,0) circle [radius=0.25];
            \draw (2,0) circle [radius=0.4];
            \draw [-latex] (-0.25,0) -- (1.6,0) node[midway, above] {$\varepsilon$};
            \draw [-latex] (-2,0) -- (-0.75,0) node[midway, above] {start};
        \end{tikzpicture}
    \end{center}
    Here, $i$ is a new state, the start state of this NFA, and $f$ is another new state, the accepting state for the NFA.

    For any subexpression $a$ in $\Sigma$, construct the NFA
    \begin{center}
        \begin{tikzpicture}
            \node (node1) at (-0.5,0) {$i$};
            \node (node2) at (2,0) {$f$};
            \draw (-0.5,0) circle [radius=0.25];
            \draw (2,0) circle [radius=0.25];
            \draw (2,0) circle [radius=0.4];
            \draw [-latex] (-0.25,0) -- (1.6,0) node[midway, above] {a};
            \draw [-latex] (-2,0) -- (-0.75,0) node[midway, above] {start};
        \end{tikzpicture}
    \end{center}
    where again $i$ and $f$ are new states, the start and accepting states, respectively. Note that in both of the basis constructions, we construct a distinct NFA, with new states, for every occurrence of $\epsilon$ or some $a$ as a subexpression of $r$.

    \noindent{\small\textbf{INDUCTION:}} Suppose $N(s)$ and $N(t)$ are NFA's for regular expression $s$ and $t$, respectively.
    \begin{itemize}
        \item[a)] Suppose $r=s|t$. Then $N(r)$, the NFA for $r$, is constructed as in Fig.\;\ref{Figure:3.40}. Here, $i$ and $f$ are new states, the start and accepting states of $N(r)$, respectively. There are $\epsilon$-transitions from $i$ to the start states of $N(s)$ and $N(t)$, and each of their accepting states have $\epsilon$-transitions to the accepting state $f$. Note that the accepting states of $N(s)$ and $N(t)$ are not accepting in $N(r)$. Since any path from $i$ to $f$ must pass through either $N(s)$ or $N(t)$ exclusively, and since the label of that path is not changed by the $\epsilon$'s leaving $i$ or entering $f$, we conclude that $N(r)$ accepts $L(s)\cup L(t)$, which is the same as $L(r)$. That is, Fig.\;\ref{Figure:3.40} is a correct construction for the union operator.
        \item[b)] Suppose $r=st$. Then construct $N(r)$ as in Fig.\;\ref{Figure:3.41}. The start state of $N(s)$ becomes the start state of $N(r)$, and the accepting state of $N(t)$ is the only accepting state of $N(r)$. The accepting state of $N(s)$ and the start state of $N(t)$ are merged into a single state, with all the transitions in or out of either state. A path from $i$ to $f$ in Fig.\;\ref{Figure:3.41} must go first through $N(s)$, and therefore its label will begin with some string in $L(s)$. The path then continues through $N(t)$, so the path's label finishes with a string in $L(t)$. Accepting states never have edges out and start states never have edges in, so it is not possible for a path to re-enter $N(s)$ after leaving it. Thus $N(r)$ accepts exactly $L(s)L(t)$, and it is a correct NFA for $r=st$.
        \item[c)] Suppose $r=s^*$. Then for $r$ we construct the NFA $N(r)$ shown in Fig.\;\ref{Figure:3.42}. Here, $i$ and $f$ are new states, the start state and lone accepting state of $N(r)$. To get from $i$ to $f$, we can either follow the introduced path labeled $\epsilon$, which takes care of the one string in $L(s)^0$, or we can go to the start state of $N(s)$, through that NFA, then from its accepting state back to its start state zero or more times. These options allow $N(r)$ to accept all the strings in $L(s)^1$, $L(s)^2$, and so on, so the entire set of strings accepted by $N(r)$ is $L(s^*)$.
        \item[d)] Finally, suppose $r=(s)$. Then $L(r)=L(s)$, and we can use the NFA $N(s)$ as $N(r)$.
    \end{itemize}
}

\begin{figure}[htbp]
    \centering
    \begin{tikzpicture}
        \node (node1) at (0,0) {$i$};
        \node (node2) at (4,0) {$f$};
        \node (node3) at (2,1) {$N(s)$};
        \node (node4) at (2,-1) {$N(t)$};
        \draw (0,0) circle [radius=0.25];
        \draw (4,0) circle [radius=0.25];
        \draw (4,0) circle [radius=0.4];
        \draw (1,1) circle [radius=0.25];
        \draw (1,-1) circle [radius=0.25];
        \draw [-latex] ({0.25*cos(45)},{0.25*sin(45)}) to node[midway,above left] {$\varepsilon$} ({1-0.25*cos(45)},{1-0.25*sin(45)});
        \draw [-latex] ({0.25*cos(45)},{-0.25*sin(45)}) to node[midway, below left] {$\varepsilon$} ({1-0.25*cos(45)},{-1+0.25*sin(45)});
        \draw (2,1) ellipse [x radius=1.5, y radius=0.5];
        \draw (2,-1) ellipse [x radius=1.5, y radius=0.5];
        \draw (3,1) circle [radius=0.25];
        \draw (3,-1) circle [radius=0.25];
        \draw [-latex] (-1.25,0) -- (-0.25,0) node[midway, above] {start};
        \draw [-latex] ({3+0.25*cos(45)},{1-0.25*sin(45)}) to node[midway, above right] {$\varepsilon$} ({4-0.4*cos(45)},{0.4*sin(45)});
        \draw [-latex] ({3+0.25*cos(45)},{-1+0.25*sin(45)}) to node[midway, below right] {$\varepsilon$} ({4-0.4*cos(45)},{-0.4*sin(45)});
    \end{tikzpicture}
    \caption{NFA for the union of two regular expressions}
    \label{Figure:3.40}
\end{figure}

\begin{figure}[htbp]
    \centering
    \begin{tikzpicture}
        \node (node1) at (0,0) {$i$};
        \node (node2) at (1,0) {$N(s)$};
        \node (node3) at (3,0) {$N(t)$};
        \node (node4) at (4,0) {$f$};
        \draw (0,0) circle [radius=0.25];
        \draw [-latex] (-1.5,0) to node[midway, above] {start} (-0.5,0);
        \draw (1,0) ellipse [x radius=1.5, y radius=0.5];
        \draw (2,0) circle [radius=0.25];
        \draw (3,0) ellipse [x radius=1.5, y radius=0.5];
        \draw (4,0) circle [radius=0.25];
        \draw (4,0) circle [radius=0.35];
    \end{tikzpicture}
    \caption{NFA for the concatenation of two regular expressions}
    \label{Figure:3.41}
\end{figure}

\begin{figure}[htbp]
    \centering
    \begin{tikzpicture}
        \node (node1) at (0,0) {$i$};
        \node (node2) at (2.25,0) {$N(s)$};
        \node (node3) at (4.5,0) {$f$};
        \draw (0,0) circle [radius=0.25];
        \draw [-latex] (-1.25,0) to node[midway, above] {start} (-0.25,0);
        \draw (1.5,0) circle [radius=0.25];
        \draw (3,0) circle [radius=0.25];
        \draw (4.5,0) circle [radius=0.25];
        \draw (4.5,0) circle [radius=0.4];
        \draw (2.25,0) ellipse [x radius=1.25, y radius=0.5];
        \draw [-latex] (0.25,0) to node[midway, above] {$\varepsilon$} (1.25,0);
        \draw [-latex] (3.25,0) to node[midway, above] {$\varepsilon$} (4.1,0);
        \draw [-latex, bend right=60] (3,0.25) to node[midway, above] {$\varepsilon$} (1.5,0.25);
        \draw [-latex, bend right=45] ({0.25*cos(45)},{-0.25*sin(45)}) to node[midway, below] {$\varepsilon$} ({4.5-0.4*cos(45)},{-0.4*sin(45)});
    \end{tikzpicture}
    \caption{NFA for the closure of a regular expression}
    \label{Figure:3.42}
\end{figure}

\section{Design of a Lexical-Analyzer Generator}
\subsection{DFA's for Lexical Analyzers}

We simulate the DFA until at some point there is no next state (or strictly speaking, the next state is $\varnothing$, the \textit{dead state} corresponding to the empty set of NFA states).

\section{Optimization of DFA-Based Pattern Matchers}
\subsection{Important States of an NFA}

We call a state of an NFA \textit{important} if it has a non-$\epsilon$ out-transition.

By concatenating a unique right endmarker $\#$ to a regular expression $r$, we give the accepting state for $r$ a transition on $\#$, making it an important state of the NFA for $(r)\#$. In other words, by using the \textit{augmented} regular expression $(r)\#$, we can forget about accepting states as the subset construction proceeds; when the construction is complete, any state with a transition on $\#$ must be an accepting state.

It is useful to present the regular expression by its \textit{syntax tree}, where the leaves correspond to operands and the interior ndes correspond to operators. An interior node is called a \textit{cat-node}, \textit{or-node}, or \textit{star-node} if it is labeled by the concatenation operator (dot), union operator |, or star operator $*$, respectively.

Leaves in a syntax tree are labeled by $\epsilon$ or by an alphabet symbol. To each leaf not labeled $\epsilon$, we attach a unique integer. We refer to this integer as the \textit{position} of the leaf and also as a position of its symbol.

\subsection{Converting a Regular Expression Directly to a DFA}

\exm{Construction of a DFA from a regular expression $r$.}{
    \noindent{\small\textbf{INPUT:}} A regular expression $r$.

    \noindent{\small\textbf{OUTPUT:}} A DFA $D$ that recognizes $L(r)$.

    \noindent{\small\textbf{METHOD:}}
    \begin{enumerate}
        \item Construct a syntax tree $T$ from the augmented regular expression $(r)\#$.
        \item Compute \textit{nullable}, \textit{firstpos}, \textit{lastpos}, and \textit{followpos} for $T$.
        \item Construct \textit{Dstates}, the set of states of DFA $D$, and \textit{Dtran}, the transition function for $D$, by the procedure of Fig.\;\ref{Figure:3.62}. The states of $D$ are sets of positions in $T$. Initially, each state is "unmarked," and a state becomes "marked" just before we consider its out-transitions. The start state of $D$ is $firstpos(n_0)$, where node $n_0$ is the root of $T$. The accepting states are those containing the position for the endmarker symbol $\#$.
    \end{enumerate}
}

\begin{center}
    \begin{tabular}{l}
        initialize \textit{Dstates} to contain only the unmarked state $firstpos(n_0)$,\\
        \qquad where $n_0$ is the root of syntax tree $T$ for $(r)\#$;\\
        \textbf{while} ( there is an unmarked state $S$ in \textit{Dstates} ) \{\\
        \qquad mark $S$;\\
        \qquad\textbf{for} ( each input symbol $a$ ) \{\\
        \qquad\qquad let $U$ be the union of $followpos(p)$ for all $p$\\
        \qquad\qquad\qquad in $S$ that correspond to $a$;\\
        \qquad\qquad\textbf{if} ( $U$ is not in \textit{Dstates} )\\
        \qquad\qquad\qquad add $U$ as an unmarked state to \textit{Dstates};\\
        \qquad\qquad$Dtran[S,a]=U$;
        \qquad\}\\
        \}
    \end{tabular}
\end{center}
\begin{figure}[htbp]
    \caption{Construction of a DFA directly from a regular expression}
    \label{Figure:3.62}
\end{figure}

\subsection{Minimizing the Number of States of a DFA}

We shall say that two automata are \textit{the same up to the state names} if one can be transformed into the other by doing nothing more than changing the names of states.

We say that string $x$ \textit{distinguishes} state $s$ from state $t$ if exactly one of the states reached from $s$ and $t$ by following the path with label $x$ is an accepting state. State $s$ is \textit{distinguishable} from state $t$ if there is some string that distinguishes them.

\exm{Minimizing the number of states of a DFA.}{
    \noindent{\small\textbf{INPUT:}} A DFA $D$ with set of states $S$, input alphabet $\Sigma$, start state $s_0$, and set of accepting states $F$.

    \noindent{\small\textbf{OUTPUT:}} A DFA $D'$ accepting the same language as $D$ and having as few states as possible.

    \noindent{\small\textbf{METHOD:}}
    \begin{enumerate}
        \item Start with an initial partition $\Pi$ with two groups, $F$ and $S-F$, the accepting and nonaccepting states of $D$.
        \item Apply the procedure of Fig.\;\ref{Figure:3.64} to construct a new partition $\Pi_{\text{new}}$.
        \item If $\Pi_{\text{new}}=\Pi$, let $\Pi_{\text{final}}=\Pi$ and continue with step (4). Otherwise, repeat step (2) with $\Pi_{\text{new}}$ in place of $\Pi$.
        \item Choose one state in each group of $\Pi_{\text{final}}$ as the \textit{representative} for that group. The representatives will be the states of the minimum-state DFA $D'$. The other components of $D'$ are constructed as follows:
        \begin{itemize}
            \item[(a)]The start state of $D'$ is the representative of the group containing the start state of $D$.
            \item[(b)]The accepting states of $D'$ are the representatives of those groups that contain an accepting state of $D$. Note that each group contains either only accepting states, or only nonaccepting states, because we started by separating those two classes of states, and the procedure of Fig.\;\ref{Figure:3.64} always forms new groups that are subgroups of previously constructed groups.
            \item[(c)]Let $s$ be the representative of some group $G$ of $\Pi_{\text{final}}$, and let the transition of $D$ from $s$ on input $a$ be to state $t$. Let $r$ be the representative of $t$'s group $H$. Then in $D'$, there is a transition from $s$ to $r$ on input $a$. Note that in $D$, every state in group $G$ must go to some state of group $H$ on input $a$, or else, group $G$ would have been split according to Fig.\;\ref{Figure:3.64}.
        \end{itemize}
    \end{enumerate}
}

\begin{center}
    \begin{tabular}{l}
        initially, let $\Pi_{\text{new}}=\Pi$;\\
        \textbf{for} ( each group $G$ of $\Pi$ ) \{\\
        \qquad partition $G$ into subgroups such that two states $s$ and $t$\\
        \qquad\qquad are in the same subgroup if and only if for all input symbols $a$, states $s$ and $t$ have transitions on $a$\\
        \qquad\qquad to states in the same group of $\Pi$;\\
        \qquad/* at worst, a state will be in a subgroup by itself */\\
        \qquad replace $G$ in $\Pi_{\text{new}}$ by the set of all subgroups formed;\\
        \}
    \end{tabular}
\end{center}
\begin{figure}[htbp]
    \caption{Construction of $\Pi_{\text{new}}$}
    \label{Figure:3.64}
\end{figure}

\chapter{Syntax Analysis}
\section{Introduction}
\subsection{Representative Grammars}

Associativity and precedence are captured in the following grammar. $E$ represents expressions consisting of terms separated by \verb|+| signs, $T$ represents terms consisting of factors separated by \verb|*| signs, and $F$ represents factors that can be either parenthesized expressions or identifiers:
\begin{equation}
    \begin{array}{ccc}E&\rightarrow&E+T\;|\;T\\T&\rightarrow&T*F\;|\;F\\F&\rightarrow&(\;E\;)\;|\;\textbf{id}\end{array}
    \label{4.1}
\end{equation}

The following non-left-recursive variant of the expression grammar (\ref{4.1}) will be used for top-down parsing:
\begin{equation}
    \begin{array}{ccl}E&\rightarrow&TE'\\E'&\rightarrow&+TE'\;|\;\epsilon\\T&\rightarrow&FT'\\T'&\rightarrow&*FT'\;|\;\epsilon\\F&\rightarrow&(\;E\;)\;|\;\textbf{id}\end{array}
    \label{4.2}
\end{equation}
The following grammar treats $+$ and $*$ alike, so it is useful for illustrating techniques for handling ambiguities during parsing:
\begin{equation}
    \begin{array}{ccc}E&\rightarrow&E+E\;|\;E*E\;|\;(\;E\;)\;|\;\textbf{id}\end{array}
    \label{4.3}
\end{equation}

\subsection{Syntax Error Handling}

Common programming errors can occur at many different levels.
\begin{itemize}
    \item\textit{Lexical} errors include misspelling of identifiers, keywords, or operators and missing quotes around text intended as a string.
    \item\textit{Syntactic} errors include misplaced semicolons or extra missing braces; that is, "\verb|{|" or "\verb|}|".
    \item\textit{Semantic} errors include type mismatches between operators and operands.
    \item\textit{Logical} errors can be anything from incorrect reasoning on the part of the programmer to the use in a C program of the assignment operator \verb|=| instead of the comparison operator \verb|==|.
\end{itemize}

Several parsing methods detect an error as soon as possible; that is, when the stream of tokens from the lexical analyzer cannot be parsed further according to the grammar for the language. More precisely, they have the \textit{viable-prefix property}, meaning that they detect that an error has occurred as soon as they see a prefix of the input that cannot be completed to form a string in the language.

\subsection{Error-Recovery Strategies}
\subsubsection{Panic-Mode Recovery}

With this method, on discovering an error, the parser discards input symbols one at a time until one of a designated set of \textit{synchronizing tokens} is found.

\section{Context-Free Grammars}
\subsection{The Formal Definition of a Context-Free Grammar}

A context-free grammar (grammar for short) consists of terminals, nonterminals, a start symbol, and productions.
\begin{enumerate}
    \item\textit{Terminals} are the basic symbols from which strings are formed.
    \item\textit{Nonterminals} are syntactic variables that denote sets of strings.
    \item In a grammar, one nonterminal is distinguished as the \textit{start symbol}, and the set of strings it denotes is the language generated by the grammar.
    \item Each \textit{production} consists of:
    \begin{itemize}
        \item[(a)] A nonterminal called the \textit{head} or \textit{left side} of the production; this production defines some of the strings denoted by the head.
        \item[(b)] The symbol $\rightarrow$.
        \item[(c)] A \textit{body} or \textit{right side} consisting of zero or more terminals and nonterminals.
    \end{itemize}
\end{enumerate}

\subsection{Derivations}

Bottom-up parsing is related to a class of derivations known as "rightmost" derivations, in which the rightmost nonterminal is rewritten at each step.

For example, consider the following grammar, with a single nonterminal $E$, which adds a production $E\to-E$ to the grammar (\ref{4.3}):
\begin{equation}
    \begin{array}{ccc}E&\rightarrow&E+E\;|\;E*E\;|\;-E\;|\;(\;E\;)\;|\;\textbf{id}\end{array}
    \label{4.7}
\end{equation}
We can take a single $E$ and repeatedly apply productions in any order to get a sequence of replacements. For example, $$E\Rightarrow-E\Rightarrow-(E)\Rightarrow-(\textbf{id})$$ We call such a sequence of replacements a \textit{derivation} of $-(\textbf{id})$ from $E$.

When a sequence of derivation steps $\alpha_1\Rightarrow\alpha_2\Rightarrow\cdots\Rightarrow\alpha_n$ rewrites $\alpha_1$ to $\alpha_n$, we say $\alpha_1$ \textit{derives} $\alpha_n$.

If $S\overset{*}{\Rightarrow}\alpha$, where $S$ is the start symbol of a grammar $G$, we say that $\alpha$ is a \textit{sentential form} of $G$. A \textit{sentence} of $G$ is a sentential form with no nonterminals. The \textit{language generated by} a grammar is its set of sentences. A language that can be generated by a grammar is said to be a \textit{context-free language}. If two grammars generate the same language, the grammars are said to be \textit{equivalent}.

To understand how parsers work, we shall consider derivations in which the nonterminal to be replaced at each step is chosen as follows:
\begin{enumerate}
    \item In \textit{leftmost} derivations, the leftmost nonterminal in each sentential is always chosen.
    \item In \textit{rightmost} derivations, the rightmost nonterminal is always chosen; we write $\alpha\underset{rm}{\Rightarrow}\beta$ in this case.
\end{enumerate}

If $S\overset{*}{\underset{lm}{\Rightarrow}}\alpha$, then we say that $\alpha$ is a \textit{left-sentential form} of the grammar at hand.

Rightmost derivations are sometimes called \textit{canonical} derivations.

\subsection{Parse Trees and Derivations}

The leaves of a parse tree are labeled by nonterminals or terminals and, read from left to right, constitute a sentential form, called the \textit{yield} or \textit{frontier} of the tree.

\subsection{Ambiguity}

A grammar that produces more than one parse tree for some sentence is said to be \textit{ambiguous}.

For most parsers, it is desirable that the grammar be made unambiguous, for if it is not, we cannot uniquely determine which parse tree to select for a sentence. In other cases, it is convenient to use carefully chosen ambiguous grammars, together with \textit{disambiguating rules} that "throw away" undesirable parse trees, leaving only one tree for each sentence.

\section{Writing a Grammar}
\subsection{Elimination of Left Recursion}

A grammar is \textit{left recursive} if it has a nonterminal $A$ such that there is a derivation $A\overset{+}{\Rightarrow}A\alpha$ for some string $\alpha$.

\exm{Eliminating left recursion.}{
\noindent{\small \textbf{INPUT:}} Grammar $G$ with no cycles or $\epsilon$-productions.

\noindent{\small \textbf{OUTPUT:}} An equivalent grammar with no left recursion.

\noindent{\small \textbf{METHOD:}} Apply the algorithm in Fig.\;\ref{Figure:4.11} to $G$. Note that the resulting non-left-recursive grammar may have $\epsilon$-productions.
}

\begin{center}
    \begin{tabular}{l}
        1)\qquad arrange the nonterminals in some order $A_1,A_2,\ldots,A_n$.\\
        2)\qquad \textbf{for} ( each $i$ from 1 to $n$ ) \{\\
        3)\qquad\qquad \textbf{for} ( each $j$ from 1 to $i-1$ ) \{\\
        4)\qquad\qquad\qquad replace each production of the form $A_i\rightarrow A_j\gamma$ by the\\
        \qquad\qquad\qquad\qquad production $A_i\rightarrow\delta_1\gamma|\delta_2\gamma|\cdots|\delta_k\gamma$, where\\
        \qquad\qquad\qquad\qquad $A_j\rightarrow\delta_1|\delta_2|\cdots|\delta_k$ are all current $A_j$-productions\\
        5)\qquad\qquad\}\\
        6)\qquad\qquad eliminate the immediate left recursion among the $A_i$-productions\\
        7)\qquad\}
    \end{tabular}
\end{center}
\begin{figure}[htbp]
    \caption{Algorithm to eliminate left recursion from a grammar}
    \label{Figure:4.11}
\end{figure}

\subsection{Left Factoring}

\exm{Left factoring a grammar.}{
\noindent {\small \textbf{INPUT:}} Grammar $G$.

\noindent {\small \textbf{OUTPUT:}} An equivalent left-factored grammar.

\noindent {\small \textbf{METHOD:}} For each nonterminal $A$, find the longest prefix $\alpha$ common to two or more of its alternatives. If $\alpha\ne\epsilon$ -- i.e., there is a nontrivial common prefix -- replace all of the $A$-productions $A\rightarrow\alpha\beta_1|\alpha\beta_2|\cdots|\alpha\beta_n|\gamma$, where $\gamma$ represents all alternatives that do not begin with $\alpha$, by $$\begin{aligned}A&\rightarrow\alpha A'\;\;|\;\;\gamma\\A'&\rightarrow\beta_1\;\;|\;\;\beta_2\;\;|\;\;\cdots\;\;|\;\;\beta_n\end{aligned}$$ Here $A'$ is a new nonterminal. Repeatedly apply this transformation until no two alternatives for a nonterminal have a common prefix.
}

\subsection{LL(1) Grammars}

\exm{Construction of a predictive parsing table.}{
\noindent {\small \textbf{INPUT:}} Grammar $G$.

\noindent {\small \textbf{OUTPUT:}} Parsing table $M$.

\noindent {\small \textbf{METHOD:}} For each production $A\to\alpha$ of the grammar, do the following:
\begin{enumerate}
    \item For each terminal $a$ in {\small FIRST}($A$), add $A\to\alpha$ to $M[A,a]$.
    \item If $\epsilon$ is in {\small FIRST}($\alpha$), then for each terminal $b$ in {\small FOLLOW}($A$), add $A\to\alpha$ to $M[A,b]$. If $\epsilon$ is in {\small FIRST}($\alpha$) and \$ is in {\small FOLLOW}(A), add $A\to\alpha$ to $M[A,\mathdollar]$ as well.
\end{enumerate}
If, after performing the above, there is no production at all in $M[A,a]$, then set $M[A,a]$ to \textbf{error} (which we normally represent by an empty entry in the table).
}

\subsection{Nonrecursive Predictive Parsing}

The behavior of the parser can be described in terms of its \textit{configurations}, which give the stack contents and the remaining input. The next algorithm describes how configurations are manipulated.

\exm{Table-driven predictive parsing.}{
\noindent {\small \textbf{INPUT:}} A string $w$ and a parsing table $M$ for grammar $G$.

\noindent {\small \textbf{OUTPUT:}} If $w$ is in $L(G)$, a leftmost derivation of $w$; otherwise, an error indication.

\noindent {\small \textbf{METHOD:}} Initially, the parser is in a configuration with $w\mathdollar$ in the input buffer and the startt symbol $S$ of $G$ on top of the stack, above \$. The program in Fig.\;\ref{Figure:4.20} uses the predictive parsing table $M$ to produce a predictive parse for the input.
}

\begin{center}
    \begin{tabular}{l}
        set $ip$ to point to the first symbol of $w$;\\
        set $X$ to the top stack symbol;\\
        \textbf{while} ( $X\ne\mathdollar$ ) \{ /* stack is not empty */\\
        \qquad\textbf{if} ( $X$ is $a$ ) pop the stack and advance $ip$;\\
        \qquad\textbf{else if} ( $X$ is a terminal ) \textit{error}();\\
        \qquad\textbf{else if} ( $M[X,a]$ is an error entry ) \textit{error}();\\
        \qquad\textbf{else if} ( $M[X,a]=X\rightarrow Y_1Y_2\cdots Y_k$ ) \{\\
        \qquad\qquad output the production $X\rightarrow Y_1Y_2\cdots Y_k$;\\
        \qquad\qquad pop the stack;\\
        \qquad\qquad push $Y_k,Y_{k-1},\ldots,Y_1$ onto the stack, with $Y_1$ on top;\\
        \qquad \}\\
        \qquad set $X$ to the top stack symbol;\\
        \}
    \end{tabular}
\end{center}
\begin{figure}[htbp]
    \caption{Predictive parsing algorithm}
    \label{Figure:4.20}
\end{figure}

\section{Bottom-Up Parsing}
\subsection{Reductions}

At each \textit{reduction} step, a specific substring matching the body of a production is replaced by the nonterminal at the head of that production.

Formally, if $S\underset{rm}{\overset{*}{\Rightarrow}}\alpha Aw\underset{rm}{\Rightarrow}\alpha\beta w$, then production $A\rightarrow\beta$ in the position following $\alpha$ is a \textit{handle} of $\alpha\beta w$.

\subsection{Shift-Reduce Parsing}

While the primary operations are shift and reduce, there are actually four possible actions a shift-reduce parser can make: (1) shift, (2) reduce, (3) accept, and (4) error.
\begin{enumerate}
    \item \textit{Shift}. Shift the next input symbol onto the top of the stack.
    \item \textit{Reduce}. The right end of the string to be reduced must be at the top of the stack. Locate the left end of the string within the stack and decide with what nonterminal to replace the string.
    \item \textit{Accept}. Announce successful completion of parsing.
    \item \textit{Error}. Discover a syntax error and call an error recovery routine.
\end{enumerate}

\subsection{Conflicts During Shift-Reduce Parsing}

There are context-free grammars for which shift-reduce parsing cannot be used. Every shift-reduce parser for such a grammar can reach a configuration in which the parser, knowing the entire stack contents and the next input symbol, cannot decide whether to shift or to reduce (a \textit{shift/reduce conflict}), or cannot decide which of several reductions to make (a \textit{reduce/reduce conflict}).

\section{Introduction to LR Parsing: Simple LR}
\subsection{Items and the LR(0) Automaton}

An \textit{LR(0) item} (\textit{item} for short) of a grammar $G$ is a production of $G$ with a dot at some position of the body.

One collection of sets of LR(0) items, called the \textit{canonical} LR(0) collection, provides the basis for constructing a deterministic finite automaton that is used to make parsing decisions. Such an automaton is called an \textit{LR(0) automaton}.

If $G$ is a grammar with start symbol $S$, then $G'$, the \textit{augmented grammar} for $G$, is $G$ with a new start symbol $S'$ and production $S'\rightarrow S$.

\subsubsection{Closure of Item Sets}

We divide all the sets of items of interest into two classes:
\begin{enumerate}
    \item \textit{Kernel items}: the initial term, $S'\rightarrow\cdot S$, and all items whose dots are not at the left end.
    \item \textit{Nonkernel items}: all items with their dots at the left end, except for $S'\rightarrow\cdot S$.
\end{enumerate}

\subsection{The LR-Parsing Algortihm}

Where a shift-reduce parser would shift a symbol, an LR parser shifts a \textit{state}.

\subsubsection{LR-Parser Configurations}

A \textit{configuration} of an LR parser is a pair: $$(s_0s_1\cdots s_m,a_ia_{i+1}\cdots a_n\$)$$ where the first component is the stack contents (top on the right), and the second component is the remaining input.

\subsubsection{Behavior of the LR Parser}

\exm{LR-parsing algorithm}{
    \noindent {\small\textbf{INPUT:}} An input string $w$ and an LR-parsing table with functions ACTION and GOTO for a grammar $G$.

    \noindent {\small\textbf{OUTPUT:}} If $w$ is in $L(G)$, the reduction steps of a bottom-up parse for $w$; otherwise, an error indication.

    \noindent {\small\textbf{METHOD:}} Initially, the parset has $s_0$ on its stack, where $s_0$ is the initial state, and $w\$$ in the input buffer. The parser then executes the program in Fig.\;\ref{Figure:4.36}.
}

\begin{center}
    \begin{tabular}{l}
        let $a$ be the first symbol of $w\$$;\\
        \textbf{while} (1) \{ /* repeat forever */\\
        \qquad let $s$ be the state on top of the stack;\\
        \qquad \textbf{if} ( $\text{ACTION}[s,a]=\text{shift }t$ ) \{\\
        \qquad\qquad push $t$ onto the stack;\\
        \qquad\qquad let $a$ be the next input symbol;\\
        \qquad \} \textbf{else if} ( $\text{ACTION}[s,a]=\text{reduce }A\rightarrow\beta$ ) \{\\
        \qquad\qquad pop $|\beta|$ symbols off the stack;\\
        \qquad\qquad let state $t$ now be on top of the stack;\\
        \qquad\qquad push $\text{GOTO}[t,A]$ onto the stack;\\
        \qquad\qquad output the production $A\rightarrow\beta$;\\
        \qquad \} \textbf{else if} ( $\text{ACTION}[s,a]=\text{accept}$ ) break; /* parsing is done */\\
        \qquad \textbf{else} call error-recovery routine;\\
        \}
    \end{tabular}
\end{center}
\begin{figure}[htbp]
    \caption{LR-parsing program}
    \label{Figure:4.36}
\end{figure}

\subsection{Constructing SLR-Parsing Tables}

\exm{Constructing an SLR-parsing table.}{
    \noindent {\small\textbf{INPUT:}} An augmented grammar $G'$.

    \noindent {\small\textbf{OUTPUT:}} The SLR-parsing table functions ACTION and GOTO for $G'$.

    \noindent {\small\textbf{METHOD:}}
    \begin{enumerate}
        \item Construct $C=\{I_0,I_1,\ldots,I_n\}$, the collection of sets of LR(0) items for $G'$.
        \item State $i$ is constructed from $I_i$. The parsing actions for state $i$ are determined as follows:
        \begin{itemize}
            \item[(a)] If $[A\rightarrow\alpha\cdot a\beta]$ is in $I_i$ and $\text{GOTO}(I_i,a)=I_j$, then set $\text{ACTION}[i,a]$ to "shift $j$." Here $a$ must be a terminal.
            \item[(b)] If $[A\rightarrow\alpha\cdot]$ is in $I_i$, then set $\text{ACTION}[i,a]$ to "reduce $A\rightarrow\alpha$" for all $a$ in FOLLOW($A$); here $A$ may not be $S'$.
            \item[(c)] If $[S'\rightarrow S\cdot]$ is in $I_i$, then set $\text{ACTION}[i,\$]$ to "accept."
        \end{itemize}

        If any conflicting actions result from the above rules, we asy the grammar is not SLR(1). The algorithm fails to produce a parser in this case.
        \item The goto transitions for state $i$ are constructed for all nonterminals $A$ using the rule: If $\text{GOTO}(I_i,A)=I_j$, then $\text{GOTO}[i,A]=j$.
        \item All entries not defined by rules (2) and (3) are made "error."
        \item The initial state of the parser is the one constructed from the set of items containing $[S'\rightarrow\cdot S]$.
    \end{enumerate}
    \label{Alg:4.46}
}

The parsing table consisting of the ACTION and GOTO functions determined by Algorithm \ref{Alg:4.46} is called the \textit{SLR(1) table for $G$}. An LR parser using the SLR(1) table for $G$ is called the SLR(1) parser for $G$, and a grammar having an SLR(1) parsing table is said to be \textit{SLR(1)}.

\subsection{Viable Prefixes}

The prefixes of right sentential forms that can appear on the stack of a shift-reduce parser are called \textit{viable prefixes}.

We say item $A\rightarrow\beta_1\cdot\beta_2$ is \textit{valid} for a viable prefix $\alpha\beta_1$ is there is a derivation $S'\underset{rm}{\overset{*}{\Rightarrow}}\alpha aw\underset{rm}{\Rightarrow}\alpha\beta_1\beta_2w$.

\section{More Powerful LR Parsers}
\subsection{Canonical LR(1) Items}

Recall that in the SLR method, state $i$ calls for reduction by $A\rightarrow\alpha$ if the set of items $I_i$ contains item $[A\rightarrow\alpha\cdot]$ and $a$ is in FOLLOW($A$).

It is possible to carry more information in the state that will allow us to rule out some of these invalid reductions by $A\rightarrow\alpha$.

The extra information is incorporated into the state by redefining items to include a terminal symbol as a second component. The general form of an item becomes $[A\rightarrow\alpha\cdot\beta,a]$, where $A\rightarrow\alpha\beta$ is a production and $a$ is a terminal or the right endmarker $\$$. We call such an object an \textit{LR(1) item}. The 1 refers to the length of the second component, called the \textit{lookahead} of the item.

Formally, we say LR(1) item $[A\rightarrow\alpha\cdot\beta,a]$ is \textit{valid} for a viable prefix $\gamma$ if there is a derivation $S\underset{rm}{\overset{*}{\Rightarrow}}\delta Aw\underset{rm}{\Rightarrow}\delta\alpha\beta w$, where
\begin{enumerate}
    \item $\gamma=\delta\alpha$, and
    \item Either $a$ is the first symbol of $w$, or $w$ is $\epsilon$ and $a$ is $\$$.
\end{enumerate}

\subsection{Constructing LR(1) Sets of Items}

\exm{Construction of the sets of LR(1) items.}{
    \noindent{\small\textbf{INPUT}:} An augmented grammar $G'$.

    \noindent{\small\textbf{OUTPUT}:} The sets of LR(1) items that are the set of items valid for one or more viable prefixes of $G'$.

    \noindent{\small\textbf{METHOD}:} The procedures CLOSURE and GOTO and the main routine \textit{items} for constructing the sets of items were shown in Fig.\;\ref{Figure:4.40}.
}

\begin{center}
    \begin{tabular}{l}
        SetOfItems CLOSURE($I$) \{\\
        \qquad\textbf{repeat}\\
        \qquad\qquad\textbf{for} ( each item $[A\rightarrow\alpha\cdot B\beta,a]$ in $I$ )\\
        \qquad\qquad\qquad\textbf{for} ( each production $B\rightarrow\gamma$ in $G'$ )\\
        \qquad\qquad\qquad\qquad\textbf{for} ( each terminal $b$ in FIRST($\beta a$) )\\
        \qquad\qquad\qquad\qquad\qquad add $[B\rightarrow\cdot\gamma,b]$ to set $I$;\\
        \qquad\textbf{until} no more items are added to $I$;\\
        \qquad\textbf{return} $I$;\\
        \}\\
        \\
        SetOfItems GOTO($I,X$) \{\\
        \qquad initialize $J$ to be the empty set;\\
        \qquad\textbf{for} ( each item $[A\rightarrow\alpha\cdot X\beta,a]$ in $I$ )\\
        \qquad\qquad add item $[A\rightarrow\alpha X\cdot\beta,a]$ to set $J$;\\
        \qquad\textbf{return} CLOSURE($J$);\\
        \}\\
        \\
        \textbf{void} \textit{items}($G'$) \{\\
        \qquad initialize $C$ to CLOSURE($\{[S'\rightarrow\cdot S,\mathdollar]\}$);\\
        \qquad\textbf{repeat}\\
        \qquad\qquad\textbf{for} ( each set of items $I$ in $C$ )\\
        \qquad\qquad\qquad\textbf{for} ( each grammar symbol $X$ )\\
        \qquad\qquad\qquad\qquad\textbf{if} ( GOTO($I,X$) is not empty and not in $C$ )\\
        \qquad\qquad\qquad\qquad\qquad add GOTO($I,X$) to $C$;\\
        \qquad\textbf{until} no new sets of items are added to $C$;\\
        \}
    \end{tabular}
\end{center}
\begin{figure}[htbp]
    \caption{Sets-of-LR(1)-items construction for grammar $G'$}
    \label{Figure:4.40}
\end{figure}

\subsection{Canonical LR(1) Parsing Tables}

We now give the rules for constructing the LR(1) ACTION and GOTO functions from the sets of LR(1) items.

\exm{Construction of canonical-LR parsing tables.}{
    \noindent{\small\textbf{INPUT}:} An augmented grammar $G'$.

    \noindent{\small\textbf{OUTPUT}:} The canonical-LR parsing table functions ACTION and GOTO for $G'$.

    \noindent{\small\textbf{METHOD}:}
    \begin{enumerate}
        \item Construct $C'=\{I_0,I_1,\ldots,I_n\}$, the collection of sets of LR(1) items for $G'$.
        \item State $i$ of the parser is constructed from $I_i$. The parsing action for state $i$ is determined as follows.
        \begin{itemize}
            \item[(a)] If $[A\rightarrow\alpha\cdot a\beta,b]$ is in $I_i$ and GOTO($I_i,a$)$=I_j$, then set ACTION[$i,a$] to "shift $j$." Here $a$ must be a terminal.
            \item[(b)] If $[A\rightarrow\alpha\cdot,a]$ is in $I_i$, $A\ne S'$, then set ACTION[$i,a$] to "reduce $A\rightarrow\alpha$."
            \item[(c)] If $[S'\rightarrow S\cdot,\mathdollar]$ is in $I_i$, then set ACTION[$i,\mathdollar$] to "accept."
        \end{itemize}
        If any conflicting actions result from the above rules, we say the grammar is not LR(1). The algorithm fails to produce a parser in this case.
        \item The goto transitions for state $i$ are constructed for all nonterminals $A$ using the rule: If GOTO$(I_i,A)=I_j$, then GOTO$([i,A])=j$.
        \item All entries not defined by rules (2) and (3) are made "error."
        \item The initial state of the parser is the one constructed from the set of items containing $[S'\rightarrow\cdot S,\mathdollar]$.
    \end{enumerate}
}

The table formed from the parsing action and goto functions produced by the above algorithm is called the \textit{canonical} LR(1) parsing table. If the parsing action function has no multiply defined entries, then the given grammar is called an \textit{LR(1) grammar}.

\subsection{Constructing LALR Parsing Tables}

We now introduce our last parser construction method, the LALR (\textit{lookahead}-LR) technique.

\exm{An easy, but space-comsuming LALR table construction.}{
    \noindent{\small\textbf{INPUT}:} An augmented grammar $G'$.

    \noindent{\small\textbf{OUTPUT}:} The LALR parsing-table functions ACTION and GOTO for $G'$.

    \noindent{\small\textbf{METHOD}:}
    \begin{enumerate}
        \item Construct $C=\{I_0,I_1,\ldots,I_n\}$, the collection of sets of LR(1) items.
        \item For each core present among the set of LR(1) items, find all sets having that core, and replace these sets by their union.
        \item Let $C'=\{J_0,J_1,\ldots,J_m\}$ be the resulting sets of LR(1) items. The parsing actions for state $i$ are constructed from $J_i$ in the same manner as in Algorithm 4.56. If there is a parsing action conflict, the algorithm fails to produce a parser, and the grammar is said not to be LALR(1).
        \item The GOTO table is constructed as follows. If $J$ is the union of one or more sets of LR(1) items, that is, $J=I_1\cup I_2\cup\cdots\cup I_k$, then the cores of GOTO($I_1,X$), GOTO($I_2,X$),..., GOTO($I_k,X$) are the same, since $I_1,I_2,\ldots,I_k$ all have the same core. Let $K$ be the union of all sets of items having the same core as GOTO($I_1,X$). Then GOTO($J,X$)$=K$.
    \end{enumerate}
}

The table produced by Algorithm 4.59 is called the \textit{LALR parsing table} for $G$. If there are no parsing action conflicts, then the given grammar is said to be an \textit{LALR(1) grammar}. The collection of sets of items constructed in step (3) is called the \textit{LALR(1) collection}.

\subsection{Efficient Construction of LALR Parsing Tables}

There are two ways a lookahead $b$ can get attached to an LR(0) item $B\rightarrow\gamma\cdot\delta$ in some set of LALR(1) items $J$:
\begin{enumerate}
    \item There is a set of items $I$, with a kernel item $A\rightarrow\alpha\cdot\beta,a$, and $J=\mathrm{GOTO}(I,X)$, and the construction of $$\text{GOTO}(\text{CLOSURE}(\{[A\rightarrow\alpha\cdot\beta,a]\}),X)$$ as given in Fig.\;\ref{Figure:4.40}, contains $[B\rightarrow\gamma\cdot\delta,b]$, regardless of $a$. Such a lookahead $b$ is said to be generated \textit{spontaneously} for $B\rightarrow\gamma\cdot\delta$.
    \item As a special case, lookahead $\mathdollar$ is generated spontaneously for the item $S'\rightarrow\cdot S$ in the initial set of items.
    \item All is as in (1), but $a=b$, and $\text{GOTO}(\text{CLOSURE}(\{[A\rightarrow\alpha\cdot\beta,b]\}),X)$, as given in Fig.\;\ref{Figure:4.40}, contains $[B\rightarrow\gamma\cdot\delta,b]$ only because $A\rightarrow\alpha\cdot\beta$ has $b$ as one of its associated lookaheads. In such a case, we say that lookaheads \textit{propagate} from $A\rightarrow\alpha\cdot\beta$ in the kernel of $I$ to $B\rightarrow\gamma\cdot\delta$ in the kernel of $J$. Note that propagation does not depend on the particular lookahead symbol; either all lookaheads propagate from one item to another, or none do.
\end{enumerate}

Let \# be a symbol not in the grammar at hand. Let $A\rightarrow\alpha\cdot\beta$ be a kernel LR(0) item in set $I$. Compute, for each $X$, $J=\text{GOTO}(\text{CLOSURE}(\{[A\rightarrow\alpha\cdot\beta,\#]\}),X)$. For each kernel item in $J$, we examine its set of lookaheads. If \# is a lookahead, then lookaheads propagate to that item from $A\rightarrow\alpha\cdot\beta$. Any other lookahead is spontaneously generated. These ideas are made precise in the following algorithm, which also makes use of the fact that the only kernel items in $J$ must have $X$ immediately to the left of the dot; that is, they must be of the form $B\rightarrow\gamma X\cdot\delta$.

\exm{Determining lookaheads.}{
    \noindent{\small\textbf{INPUT}:} The kernel $K$ of a set of LR(0) items $I$ and a grammar symbol $X$.

    \noindent{\small\textbf{OUTPUT}:} The lookaheads spontaneously generated by items in $I$ for kernel items in GOTO($I,X$) and the items in $I$ from which lookaheads are propagated to kernel items in GOTO($I,X$).

    \noindent{\small\textbf{METHOD}:} The algorithm is given in Fig.\;\ref{Figure:4.45}.
}

\begin{center}
    \begin{tabular}{l}
        \textbf{for} ( each item $A\rightarrow\alpha\cdot\beta$ in $K$ ) \{\\
        \qquad $J:=\text{CLOSURE}(\{[A\rightarrow\alpha\cdot\beta,\#]\})$;\\
        \qquad\textbf{if} ( $[B\rightarrow\gamma\cdot X\delta,a]$ is in $J$, and $a$ is not \# )\\
        \qquad\qquad conclude that lookahead $a$ is generated spontaneously for item\\
        \qquad\qquad\qquad $B\rightarrow\gamma X\cdot\delta$ in $\text{GOTO}(I,X)$;\\
        \qquad\textbf{if} ( $[B\rightarrow\gamma\cdot X\delta,\#]$ is in $J$ )\\
        \qquad\qquad conclude that lookaheads propagate from $A\rightarrow\alpha\cdot\beta$ in $I$ to\\
        \qquad\qquad\qquad $B\rightarrow\gamma X\cdot\delta$ in GOTO($I,X$);\\
        \}
    \end{tabular}
\end{center}
\begin{figure}[htbp]
    \caption{Discovering propagated and spontaneous lookaheads}
    \label{Figure:4.45}
\end{figure}

The next algorithm describes one technique to propagate lookaheads to all items.

\exm{Efficient computation of the kernels of the LALR(1) collection of sets of items.}{
    \noindent{\small\textbf{INPUT}:} An augmented grammar $G'$.

    \noindent{\small\textbf{OUTPUT}:} The kernels of the LALR(1) collection of sets of items for $G'$.

    \noindent{\small\textbf{METHOD}:}
    \begin{enumerate}
        \item Construct the kernels of the sets of LR(0) items for $G$. If space is not at a premium, the simplest way is to construct the LR(0) sets of items and then remove the nonkernel items. If space is severely constrained, we may wish instead to store only the kernel items for each set, and compute GOTO for a set of items $I$ by first computing the closure of $I$.
        \item Apply Algorithm 4.62 to the kernel of each set of LR(0) items and grammar symbol $X$ to determine which lookaheads are spontaneously generated for kernel items in GOTO($I,X$), and from which items in $I$ lookaheads are propagated to kernel items in GOTO($I,X$).
        \item Initialize a table that gives, for each kernel item in each set of items, the associated lookaheads. Initially, each item has associated with it only those lookaheads that we determined in step (2) were generated spontaneously.
        \item Make repeated passes over the kernel items in all sets. When we visit an item $i$, we look up the kernel items to which $i$ propagates its lookaheads, using information tabulated in step (2). The current set of lookaheads for $i$ is added to those already associated with each of the items to which $i$ propagates its lookaheads. We continue making passes over the kernel items until no more new lookaheads are propagated.
    \end{enumerate}
}

\section{Using Ambiguous Grammars}
\subsection{Precedence and Associativity to Resolve Conflicts}

The parser for the ambiguous grammar (\ref{4.3}) will not waste time reducing by these \textit{single} productions (productions whose body consists of a single nonterminal).

\chapter{Syntax-Directed Translation}

We associate information with a language construct by attaching \textit{attributes} to the grammar symbol(s) representing the construct.

\section{Syntax-Directed Definitions}

A \textit{syntax-directed definition} (SDD) is a context-free grammar together with attributes and rules.

\subsection{Inherited and Synthesized Attributes}

We shall deal with two kinds of attributes for nonterminals:
\begin{enumerate}
    \item A \textit{synthesized attribute} for a nonterminal $A$ at a parse-tree node $N$ is defined by a semantic rule associated with the production at $N$.
    \item An \textit{inherited attribute} for a nonterminal $B$ at a parse-tree node $N$ is defined by a semantic rule associated with the production at the parent of $N$.
\end{enumerate}

An SDD that involves only synthesized attributes is called \textit{S-attributed}.

An SDD without side effects is sometimes called an \textit{attribute grammar}.

\subsection{Evaluating an SDD at the Nodes of a Parse Tree}

A parse tree, showing the value(s) of its attribute(s) is called an \textit{annotated parse tree}.

\section{Evaluation Orders for SDD's}
\subsection{Dependency Graphs}

A \textit{dependency graph} depicts the flow of information among the attribute instances in a particular parse tree; an edge from one attribute instance to another means that the value of the first is needed to compute the second.

\subsection{Ordering the Evaluation of Attributes}

If the dependency graph has an edge from node $M$ to node $N$, then the attribute corresponding to $M$ must be evaluated before the attribute of $N$. Thus, the only allowable orders of evaluation are those sequences of nodes $N_1,N_2,\ldots,N_k$ such that if there is an edge of the dependency garph from $N_i$ to $N_j$, then $i<j$. Such an ordering embeds a directed graph into a linear order, and is called a \textit{topological sort} of the graph.

\subsection{S-Attributed Definitions}

In practice, translations can be implemented using classes of SDD's that guarantee an evaluation order, since they do not permit dependency graphs with cycles.

The first class is defined as follows:
\begin{itemize}
    \item An SDD is \textit{S-attributed} if every attribute is synthesized.
\end{itemize}

\subsection{L-attributed Definitions}

The second class of SDD's is called \textit{L-attributed definitions}.

\section{Syntax-Directed Translation Schemes}

A \textit{syntax-directed translation scheme} (SDT) is a context-free grammar with program fragments embedded within production bodies. The program fragments are called \textit{semantic actions} and can appear at any position within a production body.

SDT's that can be implemented during parsing can be characterized by introducing distinct \textit{marker nonterminals} in place of each embedded action; each marker $M$ has only one production $M\rightarrow\epsilon$.

\subsection{Postfix Translation Schemes}

SDT's with all actions at the right ends of the production bodies are called \textit{postfix SDT's}.

\section{Implementing L-Attributed SDD's}

The following methods do translation by traversing a parse tree:
\begin{enumerate}
    \item \textit{Build the parse tree and annotate}.
    \item \textit{Build the parse tree, add actions, and execute the actions in preorder}.
\end{enumerate}
In this section, we discuss the following methods for translation during parsing:
\begin{itemize}
    \item [3.]\textit{Use a recursive-descent parser} with one function for each nonterminal.
    \item [4.]\textit{Generate code on the fly}, using a recursive-descent parser.
    \item [5.]\textit{Implement an SDT in conjunction with an LL-parser}.
    \item [6.]\textit{Implement an SDT in conjunction with an LR-parser}.
\end{itemize}

\subsection{On-The-Fly Code Generation}

The construction of long strings of code that are attribute values is undesirable for several reasons, including the time it could take to copy or move long strings. In common cases such as our running code-generation example, we can instead incrementally generate pieces of the code into an array or output file by executing actions in an SDT. The elements we need to make this technique work are:
\begin{enumerate}
    \item There is, for one or more nonterminals, a \textit{main} attribute.
    \item The main attributes are synthesized.
    \item The rules that evaluate the main attribute(s) ensure that
    \begin{itemize}
        \item [(a)] The main attribute in the concatenation of main attributes of nonterminals appearing in the body of the production involved, perhaps with other elements that are not main attributes.
        \item [(b)] The main attributes of nonterminals appear in the rule in the same order as the nonterminals themselves appear in the production body.
    \end{itemize}
\end{enumerate}

\subsection{L-Attributed SDD's and LL Parsing}

In addition to records representing terminals and nonterminals, the parser stack will hold \textit{action-records} representing actions to be executed and \textit{synthesized-records} to hold the synthesized attributes for nonterminals.

\chapter{Intermediate-Code Generation}

Static checking includes \textit{type checking}, which ensures that operators are applied to compatible operands.

\section{Variants of Syntax Trees}

A directed acyclic graph (hereafter called a \textit{DAG}) for an expression identifies the \textit{common subexpressions} (subexpressions that occur more than once) of the expression.

\subsection{The Value-Number Method for Constructing DAG's}

Often, the nodes of a syntax tree or DAG are stored in an array of records.

In this array, we refer to nodes by giving the integer index of the record for that node within the array. This integer historically has been called the \textit{value number} for the node or for the expression represented by the node.

Let the \textit{signature} of an interior node be the triple $\langle op,l,r\rangle$, where $op$ is the label, $l$ its left child's value number, and $r$ its right child's value number.
\exm{The value-number method for constructing the nodes of a DAG.}{
    \noindent{\small\textbf{INPUT}:} Label $op$, node $l$, and node $r$.

    \noindent{\small\textbf{OUTPUT}:} The value number of a node in the array with signature $\langle op,l,r\rangle$.

    \noindent{\small\textbf{METHOD}:} Search the array for a node $M$ with label $op$, left child $l$, and right child $r$. If there is such a node, return the value number of $M$. If not, create in the array a new node $N$ with label $op$, left child $l$, and right child $r$, and return its value number.
}

The hash table is one of several data structures that support \textit{dictionaries} efficiently.

To construct a hash table for the nodes of a DAG, we need a \textit{hash function} $h$ that computes the index of the bucket for a signature $\langle op,l,r\rangle$, in a way that distributes the signatures across buckets, so that it is unlikely that any one bucket will get much more than a fair share of the nodes.

An array, indexed by hash value, holds the \textit{bucket headers}, each of which points to the first cell of the list.

\section{Three-Address Code}
\subsection{Addresses and Instructions}

An address can be one of the following:
\begin{itemize}
    \item A \textit{name}.
    \item A \textit{constant}.
    \item A \textit{compiler-generated temporary}.
\end{itemize}

Here is a list of the common three-address instruction forms:
\begin{enumerate}
    \item Assignment instructions of the form $x=y\;op\;z$, where $op$ is a binary arithmetic or logical operation, and $x,y$, and $z$ are addresses.
    \item Assignments of the form $x=op\;y$, where $op$ is a unary operation.
    \item \textit{Copy instructions} of the form $x=y$, where $x$ is assigned the value of $y$.
    \item An unconditional jump \texttt{goto} $L$.
    \item Conditional jumps of the form \texttt{if $x$ goto $L$} and \texttt{ifFalse $x$ goto $L$}.
    \item Conditional jumps which apply a relational operator to $x$ and $y$, and execute the instruction with label $L$ next if $x$ stands in relation $relop$ to $y$.
    \item Procedure calls and returns are implemented using the following instructions: \texttt{param $x$} for parameters; \texttt{call $p,n$} and \texttt{$y$ = call $p,n$} for procedure and function calls, respectively; and \texttt{return $y$}.
    \item Indexed copy instructions of the form $x=y[i]$ and $x[i]=y$.
    \item Address and pointer assignments of the form $x=\&y,x=*y$, and $*x=y$.
\end{enumerate}

\subsection{Quadruples}

A \textit{quadruple} (or just "\textit{quad}") has four fields, which we call $op,arg_1,arg_2$, and the $result$.

\subsection{Triples}

A \textit{triple} has only three fields, which we call $op,arg_1$, and $arg_2$.

\textit{Indirect triples} consist of a listing of pointers to triples, rather than a listing of triples themselves.

\subsection{Static Single-Assignment Form}

Two distinctive aspects distinguish SSA from three-address code. The first is that all assignments in SSA are to variables with distinct names; hence the term \textit{static single-assignment}.

\section{Types and Declarations}

The applications of types can be grouped under checking and translation:
\begin{itemize}
    \item \textit{Type checking} uses logical rules to reason about the behavior of a program at run time.
    \item \textit{Transition Applications}.
\end{itemize}

The actual storage for a procedure call or an object is allocated at run time, when the procedure is called or the object is created. As we examine local declarations at compile time, we can, however, lay out \textit{relative addresses}, where the relative address of a name or a component of a data structure is an offset from the start of a data area.

\subsection{Type Expressions}

Types have structure, which we shall represent using \textit{type expressions}: a type expression is either a basic type or is formed by applying an operator called a \textit{type constructor} to a type expression.

We shall use the following definition of type expressions:
\begin{itemize}
    \item A basic type is a type expression. Typical basic types for a language include \textit{boolean}, \textit{char}, \textit{integer}, \textit{float}, and \textit{void}; the latter denotes "the absence of a value".
    \item A type name is a type expression.
    \item A type expression can be formed by applying the \textit{array} type constructor to a number and a type expression.
    \item A record is a data structure with named fields. A type expression can be formed by applying the \textit{record} type constructor to the field names and their types.
    \item A type expression can be formed by using the type constructor $\rightarrow$ for function types.
    \item If $s$ and $t$ are type expressions, then their Cartesian product $s\times t$ is a type expression.
    \item Type expressions may contain variables whose values are type expressions.
\end{itemize}

\subsection{Type Equivalence}

When type expressions are represented by graphs, two types are \textit{structurally equivalent} if and only if one of the following conditions is true:
\begin{itemize}
    \item They are the same basic type.
    \item They are formed by applying the same constructor to structurally equivalent types.
    \item One is a type name that denotes the other.
\end{itemize}
If type names are treated as standing for themselves, then the first two conditions in the above definition lead to \textit{name equivalence} of type expressions.

\subsection{Storage Layout for Local Names}

\begin{framed}
    \begin{center}
        \textbf{{\large Address Alignment}}
    \end{center}
    
    The storage layout for data objects is strongly influenced by the addressing constraints of the target machine. For example, instructions to add integers may expect integers to be \textit{aligned}, that is, placed at certain positions in memory. Space left unused due to alignment considerations is referred to as \textit{padding}. When space is at a premium, a compiler may \textit{pack} data so that no padding is left; additional instructions may then need to be executed at run time to position packed data so that it can be operated on as if it were properly aligned.
\end{framed}

The \textit{width} of a type is the number of storage units needed for objects of that type.

\subsection{Sequences of Declarations}

% %--------------------------------------------------------------------------
% %         Bibliographie 
% %--------------------------------------------------------------------------
\end{document}